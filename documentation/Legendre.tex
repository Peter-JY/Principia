\documentclass[10pt, a4paper, twoside]{basestyle}
\usepackage[Mathematics]{semtex}
\usepackage{chngcntr}
\counterwithout{equation}{section}

%%%% Shorthands.

%%%% Title and authors.

\title{Explicit representation of Legendre polynomials}
\date{\printdate{2018-09-03}}
\author{Pascal~Leroy (phl)}
\begin{document}
\maketitle
\noindent
According to Wikipedia\footnote{\url{https://en.wikipedia.org/wiki/Legendre_polynomials}.}, the Legendre polynomials have the following representation in terms of monomials, which is immediate from the recursion formula\footnote{The same article has a simpler formula (third one in the "Explicit representation" section) which unfortunately is more prone to integer overflows.}:
\begin{equation*}
P_n(x) = 2^n \sum{k = 0}[n] x^k \binom{n}{k} \binom{\frac{n + k - 1}{2}}{n}
\end{equation*}
In this formula, the second binomial coefficient is actually nontrivial to evaluate because its upper index is not necessarily integral, and its lower index is greater than its upper index.  In other words, it involves the $\Gamma$ function.

First observe that, because of the parity of the Legendre polynomials, the terms where $k$ and $n$ do not have the same residual modulo 2 are 0.  Therefore the above can be rewritten as:
\begin{equation*}
P_n(x) = 2^n \sum{\substack{k = 0\\ k \equiv n \pmod 2}}[n] x^k \binom{n}{k} \binom{\frac{n + k - 1}{2}}{n}
\end{equation*}
Note that, in this formula, $\frac{n + k - 1}{2}$ is always a half-integer.  We are now going to compute the coefficient of $x^k$.  Expanding the binomial coefficients we obtain:
\begin{align*}
\binom{n}{k} \binom{\frac{n + k - 1}{2}}{n} &=
  \frac{\Factorial{n}}
       {\Factorial{k}\Factorial{\pa{n - k}}}
  \frac{\Factorial{\pa{\frac{n + k - 1}{2}}}}
       {\Factorial{n}\Factorial{\pa{\frac{n + k - 1}{2} - n}}}\\
&=
  \frac{1}
       {\Factorial{k}\Factorial{\pa{n - k}}}
  \frac{\Factorial{\pa{\frac{n + k - 1}{2}}}}
       {\Factorial{\pa{\frac{k - n - 1}{2}}}}
\end{align*}
This can be rewritten in terms of the $\Gamma$ function, using $\Gamma\of{n + 1} = \Factorial{n}$:
\begin{equation}
\label{coefficientusinggamma}
\binom{n}{k} \binom{\frac{n + k - 1}{2}}{n} =
  \frac{1}
       {\Factorial{k}\Factorial{\pa{n - k}}}
  \frac{\Gamma\of{\frac{n + k + 1}{2}}}
       {\Gamma\of{\frac{k - n + 1}{2}}}
\end{equation}
For positive half-integers, the $\Gamma$ function is given by:
\begin{equation}
\label{gammahalfinteger}
\Gamma\of{\frac{m}{2}} =
  \sqrt{\Pi}
  \frac{\DoubleFactorial{\pa{m - 2}}}{2^{\frac{m - 1}{2}}}
\end{equation}
Thus, since $n + k + 1 > 0$, we get:
\begin{equation*}
\Gamma\of{\frac{n + k + 1}{2}} =
  \sqrt{\Pi}
  \frac{\DoubleFactorial{\pa{n + k - 1}}}{2^{\frac{n + k}{2}}}
\end{equation*}
For computing the denominator of (\ref{coefficientusinggamma}), we observe that $k - n + 1 \leq 1$.  The $\Gamma$ function for a negative half-integer (or for $\frac{1}{2}$) may be computed using the reflection formula:
\begin{equation*}
\Gamma\of{\frac{m}{2}}\Gamma\of{1 - \frac{m}{2}} = \frac{\Pi}{\sin\of{\Pi\frac{m}{2}}} = \pa{-1}^{\frac{m + 1}{2}} \Pi
\end{equation*}
Thus:
\begin{align*}
\Gamma\of{\frac{k - n + 1}{2}} &= \frac{\pa{-1}^{\frac{n - k}{2}} \Pi}{\Gamma\of{1 - \frac{k - n + 1}{2}}}\\
&= \frac{\pa{-1}^{\frac{n - k}{2}} \Pi}{\Gamma\of{\frac{n - k + 1}{2}}}
\end{align*}
Using (\ref{gammahalfinteger}) to rewrite the denominator we obtain:
\begin{equation*}
\Gamma\of{\frac{k - n + 1}{2}} =
  \sqrt{\Pi}
  \frac
    {\pa{-1}^{\frac{n - k}{2}}{2^{\frac{n - k}{2}}}}
    {\DoubleFactorial{\pa{n - k - 1}}}
\end{equation*}
Putting everything together, the coefficient of $x^k$ is:
\begin{align*}
\binom{n}{k} \binom{\frac{n + k - 1}{2}}{n} &=
  \frac
    {\pa{-1}^{\frac{n - k}{2}}}
    {2^n}
  \frac
    {\DoubleFactorial{\pa{n + k - 1}}\DoubleFactorial{\pa{n - k - 1}}}
    {\Factorial{k}\Factorial{\pa{n - k}}}\\
&=
  \frac
    {\pa{-1}^{\frac{n - k}{2}}}
    {2^n}
  \frac
    {\DoubleFactorial{\pa{n + k - 1}}}
    {\Factorial{k}\DoubleFactorial{\pa{n - k}}}
\end{align*}
And finally:
\begin{equation*}
P_n(x) = \sum{\substack{k = 0\\ k \equiv n \pmod 2}}[n]
  x^k \pa{-1}^{\frac{n - k}{2}}
  \frac
    {\DoubleFactorial{\pa{n + k - 1}}}
    {\Factorial{k}\DoubleFactorial{\pa{n - k}}}
\end{equation*}
\end{document}