\documentclass[10pt, a4paper, twoside]{basestyle}

\usepackage{tikz}
\usetikzlibrary{cd}

\usepackage[Mathematics]{semtex}

%%%% Shorthands.

%%%% Title and authors.

\title{On the rotating-pulsating reference frame}
\date{\printdate{2023-01-23}}
\author{Robin~Leroy (egg)}
\begin{document}
\maketitle

\section*{Notation}
\paragraph*{Time} Unless otherwise mentioned, all variables implicitly depend on time, a one-dimensional real oriented inner product space $T$.
All equations are for those functions evaluated at time $t$.
Unless otherwise mentioned, all predicates involving variables dependent on time are
implicitly quantified for all times $t$.

For example, we may write $f\in\R$ for $\FunctionSpec f T \R$.

\paragraph*{Derivative} The notation $\deriv x f$ represents the evaluation
at $x$
of the differential $\diffd f$, where the expression $f$ is taken as a function of $x$.
Within $f$, any occurrences of $x$ represent a free variable, rather than a function of time.
The derivative with respect to time $\deriv t f$ may be written $\TimeDerivative f$.

For example, we may write
\[\derivop t \deriv x {\E^{x}} = \derivop t {\E^{x}} = \E^{x}\TimeDerivative x\text,\]
which is implicitly
\[\Evaluate{\derivop τ \pa{\Evaluate{\deriv ξ {\E^{ξ}}}{ξ=x\of{τ}}\phantom{_{ξ=x\of{τ}}}}}{τ=t} = \Evaluate{\deriv τ {\E^{x\of{τ}}}}{τ=t} = \E^{x\of{t}}\TimeDerivative x\of{t}\text.\]
This also applies for multidimensional $x$; in particular we may write
\[\Transpose{\deriv \vx {f\of{\vx}}}\text{ for }\pa{\grad f\of{t,\placeholder}}\of{t, \vx\of t}\text.\]
This notation facilitates changes of variables, which are the main focus of this document.

Note that $\diffd$ here is always the differential, not the exterior derivative; thus for vector
spaces $V$ and $W$ and $\FunctionSpec f V W$, we have $\FunctionSpec {\diffd f} {V} {\Dual V \Tensor W}$, and
$\FunctionSpec {\diffd^2 f} {V} {\Dual V \Tensor \Dual V \Tensor W}$, rather than $\diffd^2=0$.

As it is somewhat impractical to construct a notation which makes pullbacks of two-forms natural,
and as we do not perform changes of variables on curls, we eschew the exterior derivative entirely,
and merely get rid of orientation-dependent identifications by writing, for $\vv$ and $\vw$
implicitly dependent on $\vq$,
\[\pa{\operatorname{\StandardVectorSymbol{rot}}_{\vq}\vv}\vw\text{ for }
\pa{\mathbf{\nabla}\times\vv\of{t,\placeholder}}\of{\vq\of t}\times\vw\of{t,\vq\of t}\text.\]

\paragraph*{Reference frames}
Script capital letters denote reference frames.
For all frames $\mathscr F$, $Q^{\mathscr F}$ is the space of displacements
from the origin in frame $\mathscr F$ (representing positions in space),
a three-dimensional real inner product space associated with $\mathscr F$.
Variables $\vq^{\mathscr F}$ or $\vq_i^{\mathscr F}$ have values in $Q^{\mathscr F}$.

Reference frames are defined in relation to each other by invertible transformations; thus if
$\mathscr G$ is defined by $\vq^{\mathscr G} \DefineAs \vg\of{\vq^{\mathscr F}}$, a function $f$
that depends on $\vq^{\mathscr F}$ can be taken as a function that depends on $\vq^{\mathscr G}$,
and differentiated accordingly.

For example, for $f$ implicitly dependent on $\vq^{\mathscr F}$ as well as $t$, we may write\[
\deriv{\vq^{\mathscr G}}{f}=\deriv{\vq^{\mathscr F}}{f}\deriv{\vq^{\mathscr G}}{\vq^{\mathscr F}}\text,
\]
which is implicitly
\[
\diffd \pa{f\of{t,\placeholder}\Compose\vg\of{t,\placeholder}^{-1}}\of{\vq^{\mathscr G}\of{t}} =
\diffd \pa{f\of{t,\placeholder}}\of{\vg\of{t,\vq^{\mathscr G}\of{t}}} \diffd\vg\of{t,\placeholder}^{-1}\text.
\]

\section{Geometric potential}
Let $\vq^{\mathscr F}$ be a field of free-falling trajectories such that $\TimeDerivative\vq^{\mathscr F}=0$ at time $t$; the field $\SecondTimeDerivative\vq^{\mathscr F}$ is the field of \emph{geometric accelerations at rest}.

In all reference frames considered, the geometric accelerations at rest have a constant curl
throughout space, thus, at time $t$,\[
\operatorname{\StandardVectorSymbol{rot}}_{\vq^{\mathscr F}}\SecondTimeDerivative\vq^{\mathscr F} =
\matA^{\mathscr F}
\]
for some $\matA^{\mathscr F}$ which does not depend on $\vq^{\mathscr F}$.

The \emph{geometric potential} $V^{\mathscr F}$ of a frame $\mathscr F$ is defined on
$Q^{\mathscr F}$ at time $t$ by the equation
\begin{equation}
\SecondTimeDerivative\vq^{\mathscr F} =
-\Transpose{\deriv{\vq^{\mathscr F}}{V^{\mathscr F}}} +
\frac{1}{2}\matA^{\mathscr F}\vq^{\mathscr F}\text.
\end{equation}
The geometric potential $V^{\mathscr F}$ implictly depends on $\vq^{\mathscr F}$ as well as $t$.

The acceleration $\SecondTimeDerivative\vq^{\mathscr F}$ of a free-falling trajectory $\vq^{\mathscr F}$
is the \emph{geometric acceleration}, which, for a given frame, depends on time, position, and velocity.
The gradient of the geometric potential is the \emph{rotation-free geometric acceleration at rest}; it
depends on time and position. If the geometric acceleration at rest is irrotational, it is equal to the
rotation-free geometric acceleration at rest.

Note that unless the geometric acceleration at rest is irrotational, the geometric potential depends on
the choice of the origin of $\mathscr F$.

\section{Inertial frame}
Let $\mathscr I$ be an inertial frame. Then for all free-falling trajectories $\vq^{\mathscr I}$,
\begin{equation}
\SecondTimeDerivative\vq^{\mathscr I}=-\Transpose{\deriv{\vq^{\mathscr I}}{V^{\mathscr I}}}\text.
\end{equation}
\section{Rotating frame}
Consider the rotating reference frame $\mathscr R$ defined by
\begin{equation}
\vq^{\mathscr R} \DefineAs \matR \pa{\vq^{\mathscr I} - \vq_0^{\mathscr I}}\text.
\end{equation}
Velocities in $\mathscr R$ are related to velocities in $\mathscr I$ as follows:
\begin{equation}
\TimeDerivative\vq^{\mathscr R} =  -  \matR \TimeDerivative{\vq_0^{\mathscr I}} + \MatrixSymbol ω \vq^{\mathscr R} + \matR \TimeDerivative\vq^{\mathscr I}\text,
\end{equation}
where
\[
\TimeDerivative{\matR}=\MatrixSymbol ω \matR\text.
\]

Accelerations in $\mathscr R$ are related to accelerations in $\mathscr I$ as follows:
\begin{align}
\SecondTimeDerivative\vq^{\mathscr R}
&=  - \MatrixSymbol ω\matR \TimeDerivative{\vq_0^{\mathscr I}}
- \matR \SecondTimeDerivative\vq_0^{\mathscr I}
+ \TimeDerivative{\MatrixSymbol ω} \vq^{\mathscr R} +\MatrixSymbol ω \TimeDerivative\vq^{\mathscr R} +
\MatrixSymbol ω \matR \TimeDerivative\vq^{\mathscr I} +
\matR \SecondTimeDerivative\vq^{\mathscr I} \nonumber \\
&= - \MatrixSymbol ω\matR \TimeDerivative{\vq_0^{\mathscr I}}
- \matR \SecondTimeDerivative\vq_0^{\mathscr I}
+ \TimeDerivative{\MatrixSymbol ω} \vq^{\mathscr R}
+ \MatrixSymbol ω \TimeDerivative\vq^{\mathscr R}
+ \MatrixSymbol ω\pa{\TimeDerivative\vq^{\mathscr R}-\MatrixSymbol ω \vq^{\mathscr R} + \matR \TimeDerivative{\vq_0^{\mathscr I}}}
+ \matR \SecondTimeDerivative\vq^{\mathscr I}
\nonumber \\
&=  - \matR \SecondTimeDerivative\vq_0^{\mathscr I}
+ \underbrace{\TimeDerivative{\MatrixSymbol ω} \vq^{\mathscr R}}_{\text{Euler}}
+ \underbrace{2 \MatrixSymbol ω \TimeDerivative\vq^{\mathscr R}}_{\text{Coriolis}}
- \underbrace{\MatrixSymbol ω^2 \vq^{\mathscr R}}_{\mathclap{\text{centrifugal}}}
+ \matR \SecondTimeDerivative\vq^{\mathscr I}\text.
\end{align}
Observe that for a free-falling trajectory,
\[\matR\SecondTimeDerivative\vq^{\mathscr I}
=-\matR\Transpose{\deriv{\vq^{\mathscr I}}{V^{\mathscr I}}}
=-\Transpose{\pa{\deriv{\vq^{\mathscr I}}{V^{\mathscr I}}\matR^{-1}}}
=-\Transpose{\pa{\deriv{\vq^{\mathscr I}}{V}\deriv{\vq^{\mathscr R}}{\vq^{\mathscr I}}}}=-\Transpose{\deriv{\vq^{\mathscr R}}{V^{\mathscr I}}}\text.\]
At rest in $\mathscr R$, \idest, for $\TimeDerivative\vq^{\mathscr R}=0$, we have
\begin{align}
\SecondTimeDerivative\vq^{\mathscr R}
&=  - \matR \SecondTimeDerivative\vq_0^{\mathscr I}
+ \TimeDerivative{\MatrixSymbol ω} \vq^{\mathscr R}
- \MatrixSymbol ω^2 \vq^{\mathscr R}
+ \matR \SecondTimeDerivative\vq^{\mathscr I} \nonumber\\
&=
\TimeDerivative{\MatrixSymbol ω} \vq^{\mathscr R}-
\Transpose{\pa{\derivop{\vq^{\mathscr R}}\pa{
\InnerProduct{\vq^{\mathscr R}}{\matR \SecondTimeDerivative\vq_0^{\mathscr I}}
- \frac{\Transpose{\pa{\vq^{\mathscr R}}}\Transpose{\MatrixSymbol ω}{\MatrixSymbol ω \vq^{\mathscr R}}}{2}
+V^{\mathscr I}}}}\text,
\end{align}
so that the geometric potential is
\begin{equation}
V^{\mathscr R}=
\InnerProduct{\vq^{\mathscr R}}{\matR \SecondTimeDerivative\vq_0^{\mathscr I}}
- \frac{\InnerProduct{\MatrixSymbol ω \vq^{\mathscr R}}{\MatrixSymbol ω \vq^{\mathscr R}}}{2}
+V^{\mathscr I}\text.
\end{equation}
\section{Rotating-pulsating frame}
The rotating-pulsating reference frame $\mathscr P$ is defined by
\begin{equation}\vq^{\mathscr P} \DefineAs \frac{\vq^{\mathscr R}}{r}\text.\end{equation}
For velocities in $\mathscr P$, we have
\begin{equation}
\TimeDerivative\vq^{\mathscr P}
= -\frac{\TimeDerivative{r}}{r^2} \vq^{\mathscr R} + \frac{1}{r} \TimeDerivative\vq^{\mathscr R}
= -\frac{\TimeDerivative{r}}{r} \vq^{\mathscr P} + \frac{1}{r} \TimeDerivative\vq^{\mathscr R}\text.
\end{equation}
For accelerations in $\mathscr P$,
\begin{align}\SecondTimeDerivative\vq^{\mathscr P} &=
  \frac{\TimeDerivative{r}^2}{r^2} \vq^{\mathscr P}
- \frac{\SecondTimeDerivative{r}}{r} \vq^{\mathscr P}
- ω \TimeDerivative\vq^{\mathscr P}
- \frac{\TimeDerivative{r}}{r^2} \TimeDerivative\vq^{\mathscr R}
+ \frac{1}{r}\SecondTimeDerivative\vq^{\mathscr R}
\nonumber\\ &=
  \frac{\TimeDerivative{r}^2}{r^2} \vq^{\mathscr P}
- \frac{\SecondTimeDerivative{r}}{r} \vq^{\mathscr P}
- \frac{\TimeDerivative{r}}{r} \TimeDerivative\vq^{\mathscr P}
- \frac{\TimeDerivative{r}}{r} \pa{\TimeDerivative\vq^{\mathscr P}+\frac{\TimeDerivative{r}}{r} \vq^{\mathscr P}}
+ \frac{1}{r}\SecondTimeDerivative\vq^{\mathscr R}
\nonumber\\ &=
- \frac{\SecondTimeDerivative{r}}{r} \vq^{\mathscr P}
- 2\frac{\TimeDerivative{r}}{r} \TimeDerivative\vq^{\mathscr P}
+ \frac{1}{r}\SecondTimeDerivative\vq^{\mathscr R}
\\ &=
- \frac{\SecondTimeDerivative{r}}{r} \vq^{\mathscr P}
- 2\frac{\TimeDerivative{r}}{r} \TimeDerivative\vq^{\mathscr P}
+ \frac{1}{r}\pa{
\TimeDerivative{\MatrixSymbol ω} \vq^{\mathscr R}
+ 2 \MatrixSymbol ω \TimeDerivative\vq^{\mathscr R}
- \Transpose{\deriv{\vq^{\mathscr R}}{V^{\mathscr R}}}}
\nonumber\\ &=
- \frac{\SecondTimeDerivative{r}}{r} \vq^{\mathscr P}
- 2\frac{\TimeDerivative{r}}{r} \TimeDerivative\vq^{\mathscr P}
+ \TimeDerivative{\MatrixSymbol ω} \vq^{\mathscr P}
+ 2 \MatrixSymbol ω \pa{\TimeDerivative\vq^{\mathscr P}+\frac{\TimeDerivative r}{r}\vq^{\mathscr P}}
- \frac{1}{r^2}\Transpose{\deriv{\vq^{\mathscr P}}{V^{\mathscr R}}}
\nonumber\\ &=
\pa{2 \MatrixSymbol ω - 2\frac{\TimeDerivative{r}}{r}\Identity} \TimeDerivative\vq^{\mathscr P}
+ \pa{2 \frac{\TimeDerivative r}{r} \MatrixSymbol ω + \TimeDerivative{\MatrixSymbol ω}} \vq^{\mathscr P}
- \frac{\SecondTimeDerivative{r}}{r} \vq^{\mathscr P}
- \frac{1}{r^2}\Transpose{\deriv{\vq^{\mathscr P}}{V^{\mathscr R}}}
\nonumber\\ &=
\pa{2 \MatrixSymbol ω - 2\frac{\TimeDerivative{r}}{r}\Identity} \TimeDerivative\vq^{\mathscr P}
+ \pa{\underbrace{2 \frac{\TimeDerivative r}{r} \MatrixSymbol ω
+ \TimeDerivative{\MatrixSymbol ω}}_{\matA^{\mathscr P}}} \vq^{\mathscr P}
- \Transpose{\pa{\derivop{\vq^{\mathscr P}}\pa{\underbrace{
  \frac{\SecondTimeDerivative{r}\scal{\vq^{\mathscr P}}{\vq^{\mathscr P}}}{2r} 
+ \frac{V^{\mathscr R}}{r^2}}_{V^{\mathscr P}}}}}
\text.
\end{align}
\section{Rotating-pulsating frame of the Kepler problem}
Consider a system consisting of two point masses with time-independent
gravitational parameters $μ_1$ and $μ_2$, subject to Newtonian gravity.
A test mass is then subject to the potential\begin{align*}
V^{\mathscr I}&=
-\frac{μ_1}{\norm{\vq_1^{\mathscr I}-\vq^{\mathscr I}}}
-\frac{μ_2}{\norm{\vq_2^{\mathscr I}-\vq^{\mathscr I}}}
\\&=
-\frac{μ_1}{\norm{\vq_1^{\mathscr R}-\vq^{\mathscr R}}}
-\frac{μ_2}{\norm{\vq_2^{\mathscr R}-\vq^{\mathscr R}}}
\\&=
\frac{1}{r}\pa{\underbrace{
-\frac{μ_1}{\norm{\vq_1^{\mathscr P}-\vq^{\mathscr P}}}
-\frac{μ_2}{\norm{\vq_2^{\mathscr P}-\vq^{\mathscr P}}}}_{\DefinitionOf{V'}}}
\text.
\end{align*}

Let $\vq^{\mathscr I}_0$ be the barycentre,
\[
\vq^{\mathscr I}_0\DefineAs \frac{μ_1\vq_1^{\mathscr I}+μ_2\vq_2^{\mathscr I}}{μ_1+μ_2}\text.
\]
We have $\SecondTimeDerivative\vq_0^{\mathscr I}=\nullvec$.
Let $\matR$ be such that $\vq_1^{\mathscr R}$ and $\vq_2^{\mathscr R}$ both lie on the $x$-axis,
with $\vq_1^{\mathscr R}$ on the negative side, and such that
$\MatrixSymbol ω$ is in the positive $x\Exterior y$ direction.

Let $\vr\DefineAs\vq_1^{\mathscr I}-\vq_2^{\mathscr I}$, and $r\DefineAs\norm \vr$,
so that
\begin{align*}
\vq_1^{\mathscr R}&=\begin{pmatrix}-\frac{μ_2}{μ_1+μ_2}r\\0\\0\end{pmatrix}
&\vq_2^{\mathscr R}&=\begin{pmatrix}\frac{μ_1}{μ_1+μ_2}r\\0\\0\end{pmatrix}\text,
\intertext{and, in the pulsating frame,}
\vq_1^{\mathscr P}&=\begin{pmatrix}-\frac{μ_2}{μ_1+μ_2}\\0\\0\end{pmatrix}
&\vq_2^{\mathscr P}&=\begin{pmatrix}\frac{μ_1}{μ_1+μ_2}\\0\\0\end{pmatrix}\text.
\end{align*}

If the eccentricity vanishes, $\TimeDerivative {\MatrixSymbol ω}=\nullvec$,
so that the Euler force vanishes, and the geometric acceleration at rest in
$\mathscr R$ is irrotational. Further, $r$ is constant, thus so are
$\vq_1^{\mathscr R}$ and $\vq_2^{\mathscr R}$, and therefore the geometric
potential $V^{\mathscr R}$  is constant.
The critical points of $V^{\mathscr R}$ are thus fixed; they are the \emph{Lagrange points}.
However, when the eccentricity does not vanish, the Euler force appears, and the critical
points of the geometric potential are not fixed.

Observe that $ωr^2$ is the areal velocity of the Kepler problem, so that\begin{equation}
\deriv{t}{ωr^2}=\TimeDerivative ω r^2 + 2 ω\TimeDerivative r r = 0
\text{, and therefore }
\TimeDerivative ω + 2 \frac{\TimeDerivative r}{r} ω = 0\text.
\label{eqnArealAcceleration}
\end{equation}
Since the rotational axis is invariant in the Kepler problem,
\[
\matA^{\mathscr P}
= \TimeDerivative {\MatrixSymbol ω} + 2 \frac{\TimeDerivative r}{r} \MatrixSymbol ω
= \nullmat\text,
\]
\idest, the geometric acceleration at rest is irrotational: the pulsation of the reference
frame eliminates the Euler force.

Further, observe that, since $\SecondTimeDerivative \vr=-\frac{μ_1+μ_2}{r^2}\hat{\vr}$, we have\[
\SecondTimeDerivative r - r ω^2=\InnerProduct{\SecondTimeDerivative\vr}{\hat{\vr}}=-\frac{μ_1+μ_2}{r^2}\text.
\]
Consider now the geometric potential\begin{align*}
V^{\mathscr P} &=
  \frac{\SecondTimeDerivative{r}\scal{\vq^{\mathscr P}}{\vq^{\mathscr P}}}{2r} 
+ \frac{V^{\mathscr R}}{r^2}
\\ &=
  \frac{\SecondTimeDerivative{r}\scal{\vq^{\mathscr P}}{\vq^{\mathscr P}}}{2r} 
- \frac{\InnerProduct{\MatrixSymbol ω \vq^{\mathscr R}}{\MatrixSymbol ω \vq^{\mathscr R}}}{2r^2}
+\frac{V^{\mathscr I}}{r^2}
\\ &=
  \frac{\SecondTimeDerivative{r}\scal{\vq^{\mathscr P}}{\vq^{\mathscr P}}}{2r^2} 
- \frac{\InnerProduct{\MatrixSymbol ω \vq^{\mathscr P}}{\MatrixSymbol ω \vq^{\mathscr P}}}{2}
+\frac{V'}{r^3}\text.
\intertext{For $\vq^{P}$ in the $xy$ plane,}
V^{\mathscr P} &=
\pa{\frac{\SecondTimeDerivative{r}}{r} 
- ω^2}\frac{\pa{q^{\mathscr P}}^2}{2}
+\frac{V'}{r^3}
\\ &= \frac{1}{r^3}\pa{-\frac{\pa{μ_1+μ_2}\pa{q^{\mathscr P}}^2}{2} + V'}
\text.
\end{align*}
Thus, in the $xy$ plane, the geometric potential, while not constant, varies only by
multiplication by a position-independent scalar; in particular, its critical points are fixed.

\section{Reparametrization}
In the Kepler problem, while the geometric acceleration at rest is irrotational, the
component of the geometric acceleration due to velocity is not orthogonal to the velocity,
so that it affects the kinetic energy:\[\SecondTimeDerivative\vq^{\mathscr P} =
\pa{2 \MatrixSymbol ω - \underbrace{2\frac{\TimeDerivative{r}}{r}\Identity}_{\mathclap{\text{Not antisymmetric.}}}} \TimeDerivative\vq^{\mathscr P}
- \Transpose{\deriv{\vq^{\mathscr P}}{V^{\mathscr P}}}\text.
\]
Treatments of the elliptic restricted three-body problem, such as
\cite{Bennett1965} or \cite[587\psqq]{Szebehely1967},
solve this issue by changing the time coordinate in addition to rescaling space.
Indeed, observe that\begin{align*}
\deriv[2]{ν}\vq^{\mathscr P}
=\derivop{ν}\pa{\deriv{t}\vq^{\mathscr P}\deriv{ν}{t}}
&=\deriv[2]{t}\vq^{\mathscr P}\pa{\deriv{ν}{t}}^2
 +\deriv{t}\vq^{\mathscr P}\deriv[2]{ν}{t}\\
&=\deriv[2]{t}\vq^{\mathscr P}\pa{\deriv{ν}{t}}^2
 +\deriv{t}\vq^{\mathscr P}\pa{\derivop{t}\deriv{ν}{t}}\deriv{ν}{t}\\
&=\frac{\SecondTimeDerivative\vq^{\mathscr P}}{\TimeDerivative ν^2}
 +\TimeDerivative\vq^{\mathscr P}\pa{\derivop{t}\frac{1}{\TimeDerivative ν}}\frac{1}{\TimeDerivative ν}\\
&=\frac{\SecondTimeDerivative\vq^{\mathscr P}}{\TimeDerivative ν^2}
- \TimeDerivative\vq^{\mathscr P}\frac{\SecondTimeDerivative ν}{\TimeDerivative ν^2}\frac{1}{\TimeDerivative ν}\\&=\frac{1}{\TimeDerivative ν^2}\pa{\SecondTimeDerivative\vq^{\mathscr P}
 -\frac{\SecondTimeDerivative ν}{\TimeDerivative ν}\TimeDerivative\vq^{\mathscr P}}\text,
\end{align*}
so that picking $ν$ such that
$\frac{\SecondTimeDerivative ν}{\TimeDerivative ν}+2\frac{\TimeDerivative r}{r}=0$ eliminates
the acceleration along the velocity vector.
This condition is satisfied if $ν$ is the true anomaly, as in that case
$\TimeDerivative ν=ω$, so that the condition becomes (\ref{eqnArealAcceleration}).
This technique is not needed when considering solely the geometric potential.

\end{document}