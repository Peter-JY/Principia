\documentclass[10pt, a4paper, twoside]{basestyle}

\usepackage{tikz}
\usetikzlibrary{cd}

\usepackage[Mathematics]{semtex}

%%%% Shorthands.

%%%% Title and authors.

\title{On the rotating-pulsating reference frame}
\date{\printdate{2023-01-32}}
\author{Robin~Leroy (egg)}
\begin{document}
\maketitle

\section*{Notation}
\paragraph*{Time} Unless otherwise mentioned, all variables implicitly depend on time, a one-dimensional real oriented inner product space $T$.
All equations are for those functions evaluated at time $t$.
Unless otherwise mentioned, all predicates involving variables dependent on time are
implicitly quantified for all times $t$.

For example, we may write $f\in\R$ for $\FunctionSpec f T \R$.

\paragraph*{Derivative} The notation $\deriv x f$ represents the evaluation
at the (time-dependent) value $x$
of the differential $\diffd f$, where the expression $f$ is taken as a function of $x$.
Within $f$, any occurrences of $x$ represent a free variable, rather than a function of time.
The derivative with respect to time $\deriv t f$ may be written $\TimeDerivative f$.

For example, we may write
\[\derivop t \deriv x {\E^{x}} = \derivop t {\E^{x}} = \E^{x}\TimeDerivative x\text,\]
which is implicitly
\[\Evaluate{\derivop τ \pa{\Evaluate{\deriv ξ {\E^{ξ}}}{ξ=x\of{τ}}\phantom{_{ξ=x\of{τ}}}}}{τ=t} = \Evaluate{\deriv τ {\E^{x\of{τ}}}}{τ=t} = \E^{x\of{t}}\TimeDerivative x\of{t}\text.\]
This also applies for multidimensional $x$; in particular we may write
\[\Transpose{\deriv \vx {f\of{\vx}}}\text{ for }\pa{\grad f\of{t,\placeholder}}\of{t, \vx\of t}\text.\]
This notation facilitates changes of variables, which are the main focus of this document.

Note that $\diffd$ here is always the differential, not the exterior derivative; thus for vector
spaces $V$ and $W$ and $\FunctionSpec f V W$, we have $\FunctionSpec {\diffd f} {V} {\Dual V \Tensor W}$, and
$\FunctionSpec {\diffd^2 f} {V} {\Dual V \Tensor \Dual V \Tensor W}$, rather than $\diffd^2=0$.

As it is somewhat impractical to construct a notation which makes pullbacks of two-forms natural,
and as we do not perform changes of variables on curls, we eschew the exterior derivative entirely,
and merely get rid of orientation-dependent identifications by writing, for $\vv$ and $\vw$
implicitly dependent on $\vq$,
\[\pa{\operatorname{\StandardVectorSymbol{rot}}_{\vq}\vv}\vw\text{ for }
\pa{\mathbf{\nabla}\times\vv\of{t,\placeholder}}\of{\vq\of t}\times\vw\of{t,\vq\of t}\text.\]

\paragraph*{Reference frames}
Script capital letters denote reference frames.
For all frames $\mathscr F$, $Q^{\mathscr F}$ is the space of displacements
from the origin in frame $\mathscr F$ (representing positions in space),
a three-dimensional real inner product space associated with $\mathscr F$.
Variables $\vq^{\mathscr F}$ or $\vq_i^{\mathscr F}$ have values in $Q^{\mathscr F}$.

Reference frames are defined in relation to each other by invertible transformations; thus if
$\mathscr G$ is defined by $\vq^{\mathscr G} \DefineAs \vg\of{\vq^{\mathscr F}}$, a function $f$
that depends on $\vq^{\mathscr F}$ can be taken as a function that depends on $\vq^{\mathscr G}$,
and differentiated accordingly.

For example, for $f$ implicitly dependent on $\vq^{\mathscr F}$ as well as $t$, we may write\[
\deriv{\vq^{\mathscr G}}{f}=\deriv{\vq^{\mathscr F}}{f}\deriv{\vq^{\mathscr G}}{\vq^{\mathscr F}}\text,
\]
which is implicitly
\[
\diffd \pa{f\of{t,\placeholder}\Compose\vg\of{t,\placeholder}^{-1}}\of{\vq^{\mathscr G}\of{t}} =
\diffd \pa{f\of{t,\placeholder}}\of{\vg\of{t,\vq^{\mathscr G}\of{t}}} \diffd\vg\of{t,\placeholder}^{-1}\text.
\]

\section{Geometric potential}
Let $\vq^{\mathscr F}$ be a field of free-falling trajectories such that $\TimeDerivative\vq^{\mathscr F}=0$ at time $t$; the field $\SecondTimeDerivative\vq^{\mathscr F}$ is the field of \emph{geometric accelerations at rest}.

In all reference frames considered, the geometric accelerations at rest have a constant curl
throughout space, thus, at time $t$,\[
\operatorname{\StandardVectorSymbol{rot}}_{\vq^{\mathscr F}}\SecondTimeDerivative\vq^{\mathscr F} =
\matA^{\mathscr F}
\]
for some $\matA^{\mathscr F}$ which does not depend on $\SecondTimeDerivative\vq^{\mathscr F}$.

The \emph{geometric potential} $V^{\mathscr F}$ of a frame $\mathscr F$ is defined on
$Q^{\mathscr F}$ at time $t$ by the equation
\begin{equation}
\SecondTimeDerivative\vq^{\mathscr F} =
-\Transpose{\deriv{\vq^{\mathscr F}}{V^{\mathscr F}}} +
\frac{1}{2}\matA^{\mathscr F}\vq^{\mathscr F}\text.
\end{equation}
The geometric potential $V^{\mathscr F}$ implictly depends on $\vq^{\mathscr F}$ as well as $t$.

The acceleration $\SecondTimeDerivative\vq^{\mathscr F}$ of a free-falling trajectory $\vq^{\mathscr F}$
is the \emph{geometric acceleration}, which, for a given frame, depends on time, position, and velocity.
The gradient of the geometric potential is the \emph{rotation-free geometric acceleration at rest}; it
depends on time and position. If the geometric acceleration at rest is irrotational, it is equal to the
rotation-free geometric acceleration at rest.

Note that unless the geometric acceleration at rest is irrotational, the geometric potential depends on
the choice of the origin of $\mathscr F$.

\section{Inertial frame}
Let $\mathscr I$ be an inertial frame. Then for all free-falling trajectories $\vq^{\mathscr I}$,
\begin{equation}
\SecondTimeDerivative\vq^{\mathscr I}=-\Transpose{\deriv{\vq^{\mathscr I}}{V^{\mathscr I}}}\text.
\end{equation}
\section{Rotating frame}
Consider the rotating reference frame $\mathscr R$ defined by
\begin{equation}
\vq^{\mathscr R} \DefineAs \matR \pa{\vq^{\mathscr I} - \vq_0^{\mathscr I}}\text.
\end{equation}
Velocities in $\mathscr R$ are related to velocities in $\mathscr I$ as follows:
\begin{equation}
\TimeDerivative\vq^{\mathscr R} =  -  \matR \TimeDerivative{\vq_0^{\mathscr I}} + \MatrixSymbol\gW \vq^{\mathscr R} + \matR \TimeDerivative\vq^{\mathscr I}\text,
\end{equation}
where
\[
\TimeDerivative{\matR}=\MatrixSymbol\gW \matR\text.
\]

Accelerations in $\mathscr R$ are related to accelerations in $\mathscr I$ as follows:
\begin{align}
\SecondTimeDerivative\vq^{\mathscr R}
&=  - \MatrixSymbol\gW\matR \TimeDerivative{\vq_0^{\mathscr I}} - \matR \SecondTimeDerivative{\vq_0^{\mathscr I}} \TimeDerivative{\MatrixSymbol\gW} \vq^{\mathscr R} +\MatrixSymbol\gW \TimeDerivative\vq^{\mathscr R} + \MatrixSymbol\gW \TimeDerivative\vq^{\mathscr I} + \matR \SecondTimeDerivative\vq^{\mathscr I} \nonumber \\
&= - \MatrixSymbol\gW\matR \TimeDerivative{\vq_0^{\mathscr I}}
- \matR \SecondTimeDerivative{\vq_0^{\mathscr I}}
+ \TimeDerivative{\MatrixSymbol\gW} \vq^{\mathscr R}
+ \MatrixSymbol\gW \TimeDerivative\vq^{\mathscr R}
+ \MatrixSymbol\gW\pa{\TimeDerivative\vq^{\mathscr R}-\MatrixSymbol\gW \vq^{\mathscr R} + \matR \TimeDerivative{\vq_0^{\mathscr I}}}
+ \matR \SecondTimeDerivative\vq^{\mathscr I}
\nonumber \\
&=  - \matR \SecondTimeDerivative{\vq_0^{\mathscr I}}
+ \underbrace{\TimeDerivative{\MatrixSymbol\gW} \vq^{\mathscr R}}_{\text{Euler}}
+ \underbrace{2 \MatrixSymbol\gW \TimeDerivative\vq^{\mathscr R}}_{\text{Coriolis}}
- \underbrace{\MatrixSymbol\gW^2 \vq^{\mathscr R}}_{\mathclap{\text{centrifugal}}}
+ \matR \SecondTimeDerivative\vq^{\mathscr I}\text.
\end{align}
Observe that for a free-falling trajectory,
\[\matR\SecondTimeDerivative\vq^{\mathscr I}
=-\matR\Transpose{\deriv{\vq^{\mathscr I}}{V^{\mathscr I}}}
=-\Transpose{\pa{\deriv{\vq^{\mathscr I}}{V^{\mathscr I}}\matR^{-1}}}
=-\Transpose{\pa{\deriv{\vq^{\mathscr I}}{V}\deriv{\vq^{\mathscr R}}{\vq^{\mathscr I}}}}=-\Transpose{\deriv{\vq^{\mathscr R}}{V^{\mathscr I}}}\text.\]
At rest in $\mathscr R$, \idest, for $\TimeDerivative\vq^{\mathscr R}=0$, we have
\begin{align}
\SecondTimeDerivative\vq^{\mathscr R}
&=  - \matR \SecondTimeDerivative{\vq_0^{\mathscr I}}
+ \TimeDerivative{\MatrixSymbol\gW} \vq^{\mathscr R}
- \MatrixSymbol\gW^2 \vq^{\mathscr R}
+ \matR \SecondTimeDerivative\vq^{\mathscr I} \nonumber\\
&=
\TimeDerivative{\MatrixSymbol\gW} \vq^{\mathscr R}-
\Transpose{\pa{\derivop{\vq^{\mathscr R}}\pa{
\InnerProduct{\vq^{\mathscr R}}{\matR \SecondTimeDerivative{\vq_0^{\mathscr I}}}
- \frac{\Transpose{\pa{\vq^{\mathscr R}}}\Transpose{\MatrixSymbol\gW}{\MatrixSymbol\gW \vq^{\mathscr R}}}{2}
+V^{\mathscr I}}}}\text,
\end{align}
so that the geometric potential is
\begin{equation}
V^{\mathscr R}=
\InnerProduct{\vq^{\mathscr R}}{\matR \SecondTimeDerivative{\vq_0^{\mathscr I}}}
- \frac{\InnerProduct{\MatrixSymbol\gW \vq^{\mathscr R}}{\MatrixSymbol\gW \vq^{\mathscr R}}}{2}
+V\text.
\end{equation}
\section{Rotating-pulsating frame}
The rotating-pulsating reference frame $\mathscr P$ is defined by
\begin{equation}\vq^{\mathscr P} \DefineAs \frac{\vq^{\mathscr R}}{r}\text.\end{equation}
For velocities in $\mathscr P$, we have
\[
\TimeDerivative\vq^{\mathscr P}
= -\frac{1}{r^2} \TimeDerivative{r} \vq^{\mathscr R} + \frac{1}{r} \TimeDerivative\vq^{\mathscr R}
= -\frac{1}{r} \TimeDerivative{r} \vq^{\mathscr P} + \frac{1}{r} \TimeDerivative\vq^{\mathscr R}\text.
\]
For accelerations in $\mathscr P$,
\begin{align}\SecondTimeDerivative\vq^{\mathscr P} &=
  \frac{1}{r^2}\TimeDerivative{r}^2 \vq^{\mathscr P}
- \frac{1}{r}\SecondTimeDerivative{r} \vq^{\mathscr P}
- \frac{1}{r}\TimeDerivative{r} \TimeDerivative\vq^{\mathscr P}
- \frac{1}{r^2}\TimeDerivative{r} \TimeDerivative\vq^{\mathscr R}
+ \frac{1}{r}\SecondTimeDerivative\vq^{\mathscr R}
\nonumber\\ &=
  \frac{1}{r^2}\TimeDerivative{r}^2 \vq^{\mathscr P}
- \frac{1}{r}\SecondTimeDerivative{r} \vq^{\mathscr P}
- \frac{1}{r}\TimeDerivative{r} \TimeDerivative\vq^{\mathscr P}
- \frac{1}{r} \TimeDerivative{r} \pa{\TimeDerivative\vq^{\mathscr P}+\frac{1}{r} \vq^{\mathscr P}}
+ \frac{1}{r}\SecondTimeDerivative\vq^{\mathscr R}
\nonumber\\ &=
  \frac{1}{r}\SecondTimeDerivative{r} \vq^{\mathscr P}
- \frac{2\TimeDerivative{r}}{r} \TimeDerivative\vq^{\mathscr P}
+ \frac{1}{r}\SecondTimeDerivative\vq^{\mathscr R}
\\ &=
  \frac{1}{r}\SecondTimeDerivative{r} \vq^{\mathscr P}
- \frac{2\TimeDerivative{r}}{r} \TimeDerivative\vq^{\mathscr P}
+ \frac{1}{r}\pa{
\TimeDerivative{\MatrixSymbol\gW} \vq^{\mathscr R}
+ 2 \MatrixSymbol\gW \TimeDerivative\vq^{\mathscr R}
+ \Transpose{\deriv{\vq^{\mathscr R}}{V^{\mathscr R}}}}
\nonumber\\ &=
  \frac{1}{r}\SecondTimeDerivative{r} \vq^{\mathscr P}
- \frac{2\TimeDerivative{r}}{r} \TimeDerivative\vq^{\mathscr P}
+ \TimeDerivative{\MatrixSymbol\gW} \vq^{\mathscr P}
+ 2 \MatrixSymbol\gW \pa{\TimeDerivative\vq^{\mathscr P}+\frac{\TimeDerivative r}{r}\vq^{\mathscr P}}
+ \Transpose{\deriv{\vq^{\mathscr P}}{V^{\mathscr R}}}
\nonumber\\ &=
\pa{2 \MatrixSymbol\gW - \frac{2\TimeDerivative{r}}{r}\Identity} \TimeDerivative\vq^{\mathscr P}
+ \pa{2 \frac{\TimeDerivative r}{r} \MatrixSymbol\gW + \TimeDerivative{\MatrixSymbol\gW}} \vq^{\mathscr P}
+ \frac{1}{r}\SecondTimeDerivative{r} \vq^{\mathscr P}
+ \Transpose{\deriv{\vq^{\mathscr P}}{V^{\mathscr R}}}
\nonumber\\ &=
\pa{2 \MatrixSymbol\gW - \frac{2\TimeDerivative{r}}{r}\Identity} \TimeDerivative\vq^{\mathscr P}
+ \pa{2 \frac{\TimeDerivative r}{r} \MatrixSymbol\gW + \TimeDerivative{\MatrixSymbol\gW}} \vq^{\mathscr P}
- \Transpose{\pa{\derivop{\vq^{\mathscr P}}\pa{
- \frac{\SecondTimeDerivative{r}\scal{\vq^{\mathscr P}}{\vq^{\mathscr P}}}{2r} 
+ {V^{\mathscr R}}}}}
\text.
\end{align}
The geometric potential is thus
\[V^{\mathscr P}\DefineAs
-\frac{\SecondTimeDerivative{r}\scal{\vq^{\mathscr P}}{\vq^{\mathscr P}}}{2r}
+ {V^{\mathscr R}}\text.\]
\section{Rotating-pulsating frame of the Kepler problem}

\end{document}