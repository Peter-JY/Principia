\documentclass[10pt, a4paper, twoside]{basestyle}

\usepackage[backend=biber,firstinits=true,maxnames=100,style=alphabetic,maxalphanames=4,doi=true,isbn=false,url=false,eprint=true]{biblatex}
\bibliography{bibliography}

\usepackage[Mathematics]{semtex}
\usepackage{chngcntr}
\counterwithout{equation}{section}

%%%% Shorthands.
\DeclareMathOperator{\bias}{\mathit{bias}}
\newcommand{\round}[1]{\doubleSquareBrackets{#1}}
\newcommand{\roundToZero}[1]{\doubleSquareBrackets{#1}_0}
\newcommand{\hex}[1]{{_{16}}\mathrm{#1}}

%%%% Title and authors.

\title{A nearly correctly-rounded cube root}
\date{TODO(egg): \printdate{2017-03-36}}
\author{Robin~Leroy (eggrobin)}
\begin{document}
\maketitle
\noindent
This document describes the error analysis of the real cube root function \texttt{Cbrt} implemented in \texttt{numerics/cbrt.cpp}.

\part*{On root finding methods studied by Lagny, Halley, Householder, \emph{et alii}}

We start with a historical overview of a family of root-finding methods.

In \cite{FantetdeLagny1691a}, Lagny first presents the iterates\begin{align}
\FunctionBody*{a}{\frac{1}{2}a+\sqrt{\frac{1}{4}a^2+\frac{b}{3a}}}\text, \label{LagnyIrrationalCubeRootIterate}
\intertext{hereafter the \emph{irrational method}, and}
\FunctionBody*{a}{a+\frac{ab}{3a^3+b}}\text, \label{LagnyRationalCubeRootIterate}
\end{align}
the \emph{rational method}, for the computation of the cube root
$\cuberoot{a^3+b}$, mentioning the existence of similar methods for arbitrarily
higher powers.
In \cite{FantetdeLagny1691b} the above methods are again given, with an outline
of the general method for higher powers, and a mention of their applicability to
finding roots of polynomials other than $z^p-r$.

That general method is given in detail in \cite[19]{FantetdeLagny1692}.
Modernizing the notation, the general rule is as follows for finding a root of the monic
polynomial of degree $p\geq2$\[
f\of{z}\DefineAs z^p - \sum{k=1}[p-1]c_{k} z^{p-k} \DefinitionOf z^p - R\of{z}
\]
with an initial approximation $a$.

Separate the binomial expansion of $\pa{x+\frac{1}{2}a}^p$ into alternating sums of
degree $p$ and $p-1$ in $z$,\begin{align*}
S_1&\DefineAs\sum{\substack{k=0\\2\Divides k}}[p]\binom{p}{k}x^{p-k}\pa{\frac{1}{2}a}^k\\
S_2&\DefineAs\sum{\substack{k=0\\2\DoesNotDivide k}}[p]\binom{p}{k}x^{p-k}\pa{\frac{1}{2}a}^k
\end{align*}
and consider the polynomials in $x$\begin{align*}
P_{p}&\DefineAs S_1-\frac{1}{2}R\of{x+\frac{1}{2}a}\\
P_{p-1}&\DefineAs S_2-\frac{1}{2}R\of{x+\frac{1}{2}a}\text.
\end{align*}
Let $P_{n-1}$ be the remainder of the polynomial
division\footnote{While the rest of the method is a straightforward translation, this step bears some explanations; its description in \cite{FantetdeLagny1692} is
% TODO(egg): Right single quotation mark leads to incorrect spacing in French, so we use Apostrophe here.  Figure out what is going on.
\begin{quote}\textfrench{De ces deux égalitez, ou priſes ſéparément, ou comparées enſemble ſelon la methode des problêmes plus que déterminez tirez en une valeur d'$x$ rationelle, ou ſimplement d'un degré commode.}
\end{quote}
It is assumed that the reader is familiar with this ``method of more-than-determined problems''.
While the derivation of the root-finding method is described in painstaking detail in \cite{FantetdeLagny1733},
which outlines the treatment of overdetermined problems, it is perhaps this remark from \cite[494]{FantetdeLagny1697} which lays it out most clearly:
\begin{quote}\textfrench{Il n'y a rien de nouveau à remarquer ſur les Problemes plus que déterminez du quatriéme degré. La Regle générale eſt d'égaler tout à zero, \& de diviſer la plus haute équation par la moins élevée, ou l'également élevée l'une par l'autre, continuellement juſques à ce que l'on trouve le reſte ou le diviſeur le plus ſimple.}
\end{quote}}
of $P_{n+1}$ by $P_{n}$.

The iterate is $x+\frac{1}{2}a$, where $x$ is a root of $P_{2}$ for the irrational method,
and the root of $P_{1}$ for the rational method.

\begin{theorem}
The iterate of Lagny’s rational method for a polynomial $f$ of degree $p$ is\begin{equation}
z\mapsto z + \pa{p-1}\frac{\pa{1/f}^{\pa{p-2}}\of{z}}{\pa{1/f}^{\pa{p-1}}\of{z}}\label{LagnyRationalIterate}
\end{equation}
\begin{proof}
TODO(egg): The proof is left as an eggsercise to the reader.
\end{proof}
\end{theorem}

This iterate was described by Householder for an arbitrary analytic functions $f$ in \cite[169]{Householder1970} (see
equation (14), wherein one should take $g\Identically 1$, and theorem 4.4.2 which mentions this special case; see also
\cite{SebahGourdon2001} which explicitly gives that form).

For $p=2$ and $f$ an arbitrary polynomial, (\ref{LagnyRationalIterate}) is Newton's method, presented by Wallis in
\cite[338]{Wallis1685}.

For $p=3$ and $f$ an arbitrary polynomial, it is Halley's rational method, given by him in \cite[142--143]{Halley1694} in
an effort to generalize\footnote{Lagny's method is general, in that an iterate is given for any
polynomial, albeit one whose order changes with the degree. However, while he refers to its
results---and even corrects a misprint therein---, Halley
did not have access to a copy of \cite{FantetdeLagny1692},
\begin{quote}
Has Regulas, cum nondum librum videram, ab amico communicatas habui
\end{quote}
and it appears that said friend communicated only the formul\ae for the cube and fifth root, as opposed to the
general method and its proof, as Halley writes
\begin{quote}
[...] \emph{D. de Lagney} [...] qui cum totus fere sit in eliciendis Potestatum purarum radicibus,
praefertim Cubic\^a, pauca tantum eaque perplexa nec satis demonstrata de affectarum radicum
extractione subjungit.
\end{quote}
or, about Lagny's irrational method for the fifth root,
\begin{quote}
Author autem nullibi inveniendi methodum ejusve demonstrationem concedit,
etiamsi maxime desiderari videatur [...].
\end{quote}
Being unaware of this generality, Halley sets out to generalize (\ref{LagnyIrrationalCubeRootIterate}) and (\ref{LagnyRationalCubeRootIterate}) to
arbitrary polynomials, and does so by keeping the order constant.} Lagny's (\ref{LagnyRationalCubeRootIterate}).
It is generally simply known as Halley's method, as the irrational method (which likewise generalizes Lagny's irrational
method for $p=3$, while retaining constant order as the degree changes) has comparatively fallen into obscurity;
see \cite{ScavoThoo1995}.

Considering that generalizing from arbitrary polynomials to other functions does not change the method,
we call the method given by the iterate (\ref{LagnyRationalIterate})\begin{itemize}
\setlength\itemsep{0em}
\item Newton’s method when $p=2$, for arbitrary $f$;
\item Lagny’s rational method when $p>2$ and $f$ is a polynomial of degree $p$;
\item Halley’s (rational) method when $p=3$ and $f$ is not a polynomial of degree $p$;
\item the Lagny--Householder method otherwise.
\end{itemize}
Note that we avoid the name ``Householder’s method'' in this last case, as it is often used to refer to a
different third order iterate instead, see TODO CITATION.


\part*{Computing a real cube root}

We now turn to the computation in \texttt{numerics/cbrt.cpp}.

\section*{Overview}
The general approach to compute the cube root of $y>0$ is the same as the one described in \cite{KahanBindel2001}:
\begin{enumerate}
\item integer arithmetic is used to get a an initial quick approximation $q$ of $\cuberoot y$;
\item a root finding method is used to improve that that to an approximation $ξ$ with a third of the precision;
\item $ξ$ is rounded to a third of the precision, resulting in the rounded approximation $x$ whose cube $x^3$ can be computed exactly;
\item a single high order iterate of a root finding method is used to get the final result.
\end{enumerate}

\section*{Notation}
We define the fractional part as $\FractionalPart a\DefineAs a-\Floor a\in\intclop{0}{1}$, regardless of the sign of $a$.

The quantities $p\in\N$ (precision in bits) and $\bias\in\N$ are as defined in IEEE 754-2008. 

We use capital letters fixed-point numbers involved in the computation, and $A>0$ for the normal floating-point number $a>0$ reinterpreted as a binary fixed-point number with $t$ bits after the binary point\footnote{The implementation uses integers (obtained by multiplying the fixed-point numbers by $2^{p-1}$). For consistency with \cite{KahanBindel2001} we work with fixed-point numbers here. Since we do not multiply fixed point numbers together, the expressions are unchanged.},\begin{align*}
  A &\DefineAs \bias + \Floor{\log_2 a} + \FractionalPart \pa{2^{-\Floor{\log_2 a}}a}\\
    &= \bias + \Floor{\log_2 a} + 2^{-\Floor{\log_2 a}}a - 1\text,\\
\intertext{and \emph{vice versa},}
  a &\DefineAs 2^{\Floor A-\bias} \pa{1+\FractionalPart{A}}\text.\\
\end{align*}
This corresponds to \cite{KahanBindel2001}'s $B+K+F$.

For both fixed- and floating-point numbers, given $α\in\R$, we write $\round{α}$ for the nearest representable number (rounding ties to even).
For fixed-point numbers, we write $\roundToZero{α}$ for directed rounding towards $0$ to the fixed-point precision (as in division implemented with integer division).

Except in the section on rescaling, the input $y$ and all intervening floating-point numbers are taken to be normal; the rescaling performed to avoid overflows also avoids subnormals.

\section{Quick approximation}
The quick approximation $q$ is computed using fixed-point arithmetic as\[
Q\DefineAs C + \roundToZero{\frac{Y}{3}}\text,
\]
where the fixed-point constant $C$ is defined as\footnote{Note
that there is a typo in the corresponding expression $C\DefineAs\pa{B-0.1009678}/3$ in \cite{KahanBindel2001}; a factor of $2$ is missing on the bias term.}\[
C\DefineAs \round{\frac{2\bias-γ}{3}}
\]
for some $γ\in\R$.

Let $ε \DefineAs \frac{q}{\sqrt[3] y}-1$,  % TODO(egg): Figure out why \cuberoot does not render here.
so that $\cuberoot y\pa{1+ε}=q$; the relative error of $q$ as an approximation of $\cuberoot y$ is $\abs ε$.
Considering $Y$, $Q$, $q$, and $ε$ as functions of $y$, we have\begin{align*}
Y\of{8y} &= Y\of{y} + 3\text,\\
Q\of{8y} &= Q\of{y} + 1\text,\\
q\of{8y} &= 2q\of{y}\text,\\
ε\of{8y} &= ε\of{y}\text,
\end{align*}
so that the properties of $ε$ need only be studied on some interval of the form $\intclop{η}{8η}$.

Pick $η\DefineAs2^{\Floor{γ}}$, and $y\in\intclop{η}{8η}=\intclop{2^{\Floor{γ}}}{2^{\Floor{γ}+3}}$,
so that $\log_2 y\in\intclop{\Floor{γ}}{\Floor{γ}+3}$. Let $k\DefineAs\Floor{\log_2 y}-\Floor{γ}$; note that $k\in\set{0,1,2}$.
Let $f\DefineAs\FractionalPart\pa{2^{-\Floor{\log_2 y}}y}\in\intclop{0}{1}$.
Up to at most $1.5$ units in the last place from rounding,\begin{align*}
Q\approx Q'\DefineAs&\bias+\frac{\Floor{\log_2 y}}{3}+\frac{\FractionalPart\pa{2^{-\Floor{\log_2 y}}y}-γ}{3}\text,\\
=&\bias+\frac{\Floor{γ}+k}{3}+\frac{f-γ}{3}\text,\\
=&\bias+\frac{k+f-\FractionalPart γ}{3}\text.
\end{align*}
Since $k\in\intclos{0}{2}$, the numerator $k+f-\FractionalPart γ$ lies in $\intopen{-1}{3}$.
Further, it is negative only if $k=0$, so that\begin{align*}
\Floor{Q'}&=\begin{cases}
\bias - 1 & \text{if }k = 0\text{ and }\FractionalPart γ > \FractionalPart\pa{2^{-\Floor{γ}}y}\text, \\
\bias & \text{otherwise,}
\end{cases}\text{ and}\\
\FractionalPart{Q'}&=\begin{cases}
1+\frac{f-\FractionalPart γ}{3} & \text{if }k = 0\text{ and }\FractionalPart γ > f\text, \\
\frac{k+f-\FractionalPart γ}{3} & \text{otherwise.}
\end{cases}
\end{align*}
Accordingly, for the quick approximation $q$, we have, again up to at most $1.5$ units in the last place,\begin{align*}
q\approx q' &= \begin{cases}
1+\frac{f-\FractionalPart γ}{6} & \text{if }k = 0\text{ and }\FractionalPart γ > f\text, \\
1+\frac{k+f-\FractionalPart γ}{3} & \text{otherwise,}
\end{cases}
\end{align*}
With $\cuberoot y = 2^{\frac{\Floor{γ}+k}{3}}\cuberoot{1+f}$, we can define\[
ε' \DefineAs \frac{q'}{\cuberoot y}-1\text,
\]
which we can express piecewise as a function of $f$ and $k$. This gives us a bound on the relative error,\[
\abs ε \leq \abs{ε'}+1.5 \Multiply 2^{p-1}\pa{1+\abs{ε'}}\text.
\]
The values $γ=0.1009678$ and $ε<3.2\%$ from \cite{KahanBindel2001} may be recovered by choosing $γ$ minimizing the maximum of $\abs{ε'}$ over $y\in\intclop{η}{8η}$,
or equivalently.\begin{align*}
γ_{\mathrm{Kahan}}\DefineAs\argmin_{γ\in\R}\max_{y\in\intclop{η}{8η}}\abs{ε'}
&=\argmin_{γ\in\R}\max_{\tuple{f, k}}\abs{ε'}\\
\intertext{where the maximum is taken over $\tuple{f, k}\in\intclop{0}{\FractionalPart γ}\Cartesian\set{0}\Union\intclop{0}{1}\Cartesian\set{1,2}$,}
&=\argmin_{γ\in\R}\max_{\tuple{f, k}\in\mathscr{E}\Union\mathscr{L}}\abs{ε'}\text,
\end{align*}
where $\mathscr{E}\DefineAs\set{\tuple{\FractionalPart γ, 0}}\Union\setst{\tuple{0, k}}{k\in\set{0,1,2}}$ is the set of the endpoints of the intervals whereon $q'$ is piecewise affine, and
$\mathscr{L}\DefineAs\setst{\tuple{\frac{k-\FractionalPart γ}{2},k}}{k\in\set{1,2}}$ are the local extrema.

The values are more precisely\footnote{These may be computed formally, but the expressions are unwieldy.}\begin{align*}
γ_{\mathrm{Kahan}} &\approx 0.10096\,78121\,55802\,88786\,36993\,42643\,55358\,06489\,88235\,75289\\
\intertext{with}
\max_{y}\abs{ε'} &\approx 0.03155\,46327\,73624\,80606\,11789\,73328\,17135\,58940\,02093\,40816\text,
\end{align*}
leading to $C_{\mathrm{Kahan}}=\hex{2A9F\,7625\,3119\,D328}\Multiply 2^{-52}$ for IEEE 754-2008 binary64.
However, as we will see in the next section, this value does not optimize the final error, so it is not the one that we use.
\section{Getting to a third of the precision}
We use a single iterate of Lagny’s method to compute $ξ$,\[
ξ\DefineAs
\round{q - \round{\frac{\round{\pa{\round{\round{q^2}q}-y} q}}
                       {\round{2\round{\round{q^2}q}+y}}}}\text.
\]
Note that the subtraction in the numerator is exact by Sterbenz's lemma.
Let $Δ \DefineAs \frac{ξ}{\sqrt[3] y}-1$ and  % TODO(egg): Figure out why \cuberoot does not render here.
\[
ξ'=q'-\frac{\pa{{q'}^3-y}q'}{2{q'}^3+y}\text.
\]
We have, up to rounding errors (TODO: bound those),\[
Δ \approx Δ' \DefineAs \frac{ξ'}{\sqrt[3] y}-1\text.
\]
With $q'=\cuberoot{y}\pa{1+ε'}$, we can express $Δ'$ using the transformation of
the relative error error by one step of Fantet de Lagny’s method on the cube root,\[
Δ' = \frac{2{ε'}^3+{ε'}^4}{3+6ε'+6{ε'}^2+2{ε'}^3}\text.
\]
If $q'$ is computed using $γ=γ_{\mathrm{Kahan}}$, we get\begin{align*}
\max_{y} \abs{Δ'} \approx 0.00002196\text,\\
\log_2 \max_{y} \abs{Δ'} \approx -15.47\text.
\end{align*}
However, $γ_{\mathrm{Kahan}}$, which minimizes $\max_{y} \abs{ε}$, does not
minimize $\max_{y} \abs{Δ'}$. This is because while $Δ'$ is monotonic as a
function of $ε'$, it is not odd: positive errors are reduced more than negative
errors are, so that the minimum is attained for a different value of $γ$.
Specifically, we have\begin{align*}
γ_{\mathrm{L}}&\DefineAs\argmin_{γ\in\R}\max_{y}\abs{Δ'}\\
&\approx 0.09918\,74615\,29855\,99525\,66149\,20761\,31234\,34720\,23067\,92759
\intertext{with}
\max_{y} \abs{ε'} &\approx 0.03103\,20521\,29929\,93577\,08166\,75859\,02139\,33719\,41389\,93269\text,
\intertext{but}
\max_{y} \abs{Δ'} &\approx 0.00002\,08686\,35536\,39593\,48770\,92008\,39844\,10254\,14831\,61229\text.
\end{align*}
The corresponding fixed-point constant is $C_{\mathrm{L}}\DefineAs\hex{2A9F\,7893\,782D\,A1CE} \Multiply 2^{-52}$ for binary64.
\end{document}
