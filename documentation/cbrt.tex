\documentclass[10pt, a4paper, twoside]{basestyle}

\usepackage[backend=biber,firstinits=true,maxnames=100,style=alphabetic,maxalphanames=4,doi=true,isbn=false,url=false,eprint=true]{biblatex}
\bibliography{bibliography}

\usepackage[Mathematics]{semtex}
\usepackage{chngcntr}
\counterwithout{equation}{section}

%%%% Shorthands.

%%%% Title and authors.

\title{%
\textdisplay{%
Cube root%
}%
}
\author{Robin~Leroy(eggrobin)}
\begin{document}
\maketitle
\noindent
This document describes the implementation of the real cube root in `numerics/cbrt.cpp`, with an error analysis.
\end{sloppypar}

\section*{Overview and notation}
The general approach to compute the cube root of $y>0$ is the same as the one described in \cite{Kahan2001}:
\begin{enumerate}
\item integer arithmetic is used to get a an initial quick approximation $q$ of $\cbrt y$;
\item a root finding method is used to improve that that to an approximation $ξ$ with a third of the precision;
\item $ξ$ is rounded to a third of the precision, resulting in the rounded approximation $x$ whose cube $x^3$ can be computed exactly;
\item a high order iterate of a root finding method is used to get the final result.
\end{enumerate}

\printbibliography
\end{document}
