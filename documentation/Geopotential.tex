\documentclass[10pt, a4paper, twoside]{basestyle}

\bibliography{bibliography}

\usepackage[Mathematics]{semtex}
\usepackage{chngcntr}
\counterwithout{equation}{section}

%%%% Shorthands.

\newcommand{\p}{\AssociatedLegendreFunctionAstro}

%%%% Title and authors.

\title{%
\textdisplay{%
Geopotential%
}%
}
\author{Pascal~Leroy (phl) and Robin~Leroy (egg)}
\begin{document}
\maketitle
\begin{sloppypar}
\noindent
This document describes the computations that are performed by the method \texttt{GeneralSphericalHarmonicsAcceleration} of class \texttt{Geopotential} to determine the acceleration exerted by a non-spherical celestial on a point mass.
\end{sloppypar}

\subsection*{Notation}
Let $\vr$ be the vector going from the centre of the celestial to the point mass.  Let $\tuple{\hat{\vx}, \hat{\vy}, \hat{\vz}}$ be a (direct) base whose $\hat{\vz}$ axis is along the axis of rotation of the celestial and whose $\hat{\vx}$ axis points toward a reference point on the celestial.  In this base $\vr$ has coordinates $\tuple{x, y, z}$ which can be expressed in terms of the latitude $\gb \in \intclos{\frac{\Pi}{2}}{\frac{\Pi}{2}}$ and the longitude
$\gl \in \intclos{0}{2 \Pi}$:
\[
\vr=\begin{pmatrix}
x \\ y \\ z
\end{pmatrix}
=\begin{pmatrix}
r \cos \gb \cos \gl\\
r \cos \gb \sin \gl\\
r \sin \gb
\end{pmatrix}
\]
where $r$ is the norm of $\vr$.  Note that $\cos \gb > 0$, which will come handy when simplifying expressions like $\sqrt{1 - \sin^2 \gb}$.

\subsection*{Potential and acceleration}
The gravitational potential due to the celestial has the form\cite{IERSConventions2010}:
\[
U\of\vr = -\frac{\gm}{r}\pa{1 + \sum{n = 1}[\infty] \sum{m = 0}[n] 
\pa{\frac{R}{r}}^n \p{n}{m}\of{\sin \gb}
\pa{C_{n m} \cos m \gl + S_{n m} \sin m \gl}
}
\]
where $\p{n}{m}$ is the associated Legendre function defined as\cite[appendix]{RiesBettadpurEanesKangKoMcCulloughNagelPiePooleRichterSaveTapley2016}:
\[
\p{n}{m} = \pa{1 - t^2}^{\frac{m}{2}} \deriv[m] t {\LegendrePolynomial n\of t}
\]
The terms that are relevant for evaluating the acceleration exerted by the spherical harmonics on the point mass have the form:
\[
V_{n m}\of\vr = \frac{1}{r}\pa{\frac{R}{r}}^n \p{n}{m}\of{\sin \gb}
\pa{C_{n m} \cos m \gl + S_{n m} \sin m \gl}
\]
and the acceleration itself is proportional to $\grad V_{n m}\of\vr$.

$V_{n m}\of\vr$ can be written as the product of a radial term, a latitudinal term and a longitudinal term, dependent of $r$, $\gb$, and $\gl$, respectively:
\[
V_{n m}\of\vr = {\mathfrak R\of r}\,{\mathfrak B\of \gb}\,{\mathfrak L\of \gl}
\]
where:
\[
\begin{dcases}
\mathfrak R\of r &= \frac{1}{r}\pa{\frac{R}{r}}^n\\
\mathfrak B\of \gb &= \p{n}{m}\of{\sin \gb}\\
\mathfrak L\of \gl &= C_{n m} \cos m \gl + S_{n m} \sin m \gl
\end{dcases}
\]
The overall gradient can then be written as:
\[
\grad V_{n m}\of\vr = 
{\grad \mathfrak R\of r}\,{\mathfrak B\of \gb}\,{\mathfrak L\of \gl} +
{\mathfrak R\of r}\,{\grad \mathfrak B\of \gb}\,{\mathfrak L\of \gl} +
{\mathfrak R\of r}\,{\mathfrak B\of \gb}\,{\grad \mathfrak L\of \gl}
\]
\subsection*{Lemmata}
To compute the acceleration, we first determine the gradient of various elements appearing in the potential.  The following lemma determines the gradient of the $\mathfrak R$ term:
\begin{lemma}
\[
\grad r^n = n r^{n - 2} \vr\text.
\]
\begin{proof}
We have trivially:
\[
\grad r = \grad \sqrt{x^2 + y^2 + z^2} = \frac{\vr}{r}
\]
from which we deduce:
\[
\grad r^n = n r^{n - 1} \grad r = n r^{n - 2} \vr
\]
\end{proof}
\end{lemma}
The following lemma is useful for computing the gradient of the $\mathfrak B$ term:
\begin{lemma}
\[
\grad \sin \gb =\frac{\cos \gb}{r}\begin{pmatrix}
-\sin \gb \cos \gl \\
-\sin \gb \sin \gl \\
\cos \gb
\end{pmatrix}\text.
\]
\begin{proof}
Noting that $\grad z = \hat{\vz}$ we have:
\[
\grad \sin \gb = \grad \frac{z}{r}
= \frac{r \hat{\vz} - z r^{-1} \vr}{r^2} = \frac{r^2 \hat{\vz} - z \vr}{r^3}
\]
which can be written, in coordinates:
\[
\grad \sin \gb =\frac{1}{r^3}\begin{pmatrix}
-x z \\ -y z \\ x^2 + y^2
\end{pmatrix}
=\frac{1}{r^3}\begin{pmatrix}
-r^2 \sin \gb \cos \gb \cos \gl \\
-r^2 \sin \gb \cos \gb \sin \gl \\
r^2 \cos^2 \gb
\end{pmatrix}
=\frac{\cos \gb}{r}\begin{pmatrix}
-\sin \gb \cos \gl \\
-\sin \gb \sin \gl \\
\cos \gb
\end{pmatrix}
\]
\end{proof}
\end{lemma}
For the $\mathfrak L$ term, we will need to evaluate the quantities:
\[
\begin{cases}
\grad \cos m \gl &= -m \sin m \gl \, \grad \gl \\
\grad \sin m \gl &= m \cos m \gl \, \grad \gl
\end{cases}
\]
The following lemma helps with that computation:
\begin{lemma}
\[
\grad \gl = \frac{1}{r \cos \gb}\begin{pmatrix}
-\sin \gl \\
\cos \gl \\
0
\end{pmatrix}\text.
\]
\begin{proof}
The angle $\gl$ is $\arctan\frac{y}{x}$ thus:
\[
\grad \gl = \frac{1}{1 + \pa{\frac{y}{x}}^2} \grad\of{\frac{y}{x}}
= \frac{1}{1 + \pa{\frac{y}{x}}^2}\frac{x \hat{\vy} - y \hat{\vx}}{x^2}
= \frac{x \hat{\vy} - y \hat{\vx}}{x^2 + y^2}
\]
which can be written, in coordinates:
\[
\grad \gl = \frac{1}{x^2 + y^2}\begin{pmatrix}
-y \\
x \\
0
\end{pmatrix}
= \frac{1}{r^2 \cos^2 \gb}\begin{pmatrix}
-r \cos \gb \sin \gl \\
r \cos \gb \cos \gl \\
0
\end{pmatrix}
= \frac{1}{r \cos \gb}\begin{pmatrix}
-\sin \gl \\
\cos \gl \\
0
\end{pmatrix}
\]
\end{proof}
\end{lemma}
Finally, we will need the gradient of the associated Legendre polynomial, given by the following lemma:
\begin{lemma}
\[
\p{n}{m}'\of{t} 
= \pa{1 - t^2}^{\frac{m}{2}}\deriv[m + 1]{t}{\LegendrePolynomial{n}\of t} - 
m t \pa{1 - t^2}^{\frac{m - 2}{2}} \deriv[m]{t}{\LegendrePolynomial{n}\of t}\text.
\]
\begin{proof}
This follows trivially from the definition:
\[
\p{n}{m}'\of{t} = \frac{m}{2} \pa{1 - t^2}^{\frac{m}{2} - 1}
\pa{-2 t} \deriv[m]{t}{\LegendrePolynomial{n}\of t} +
\pa{1 - t^2}^{\frac{m}{2}} \deriv[m + 1]{t}{\LegendrePolynomial{n}\of t}
\]
\end{proof}
\end{lemma}
\subsection*{Gradients}
We can now compute the gradient of the three terms that make up $V_{n m}\of\vr$.  First, the radial term:
\[
\grad \mathfrak R\of r = R^n \grad r^{-\pa{n + 1}} = -\pa{n + 1} R^n r^{-\pa{n + 3}} \vr
=-\pa{n + 1}\frac{\mathfrak R\of r}{r^2} \vr
\]
For the latitudinal term, the chain rule yields:
\[
\grad \mathfrak B\of \gb = \p{n}{m}\of{\sin \gb} \, \grad \sin \gb
= \p{n}{m}'\of{\sin \gb} \frac{\cos \gb}{r}\begin{pmatrix}
-\sin \gb \cos \gl \\
-\sin \gb \sin \gl \\
\cos \gb
\end{pmatrix}
\]
Substituting $\sin \gb$ for the argument of $\p{n}{m}$ and its derivative we obtain:
\[
\begin{dcases}
\p{n}{m}\of{\sin \gb} = {\cos^m \gb}
\Evaluate{\deriv[m]{t}{\LegendrePolynomial{n}\of t}}{t = \sin \gb} \\
\p{n}{m}'\of{\sin \gb} 
= {\cos^m \gb} \Evaluate{\deriv[m + 1]{t}{\LegendrePolynomial{n}\of t}}{t = \sin \gb} -
m \sin \gb \pa{\cos \gb}^{m - 2}
\Evaluate{\deriv[m]{t}{\LegendrePolynomial{n}\of t}}{t = \sin \gb}
\end{dcases}
\]
and thus:
\[
\begin{dcases}
\mathfrak B\of \gb = {\cos^m \gb}
\Evaluate{\deriv[m]{t}{\LegendrePolynomial{n}\of t}}{t = \sin \gb} \\
\grad \mathfrak B\of \gb = \frac{1}{r}
\pa{\pa{\cos \gb}^{m + 1}
\Evaluate{\deriv[m + 1]{t}{\LegendrePolynomial{n}\of t}}{t = \sin \gb} -
m \sin \gb \pa{\cos \gb}^{m - 1}
\Evaluate{\deriv[m]{t}{\LegendrePolynomial{n}\of t}}{t = \sin \gb}\qquad\ \,}
\begin{pmatrix}
-\sin \gb \cos \gl \\
-\sin \gb \sin \gl \\
\cos \gb
\end{pmatrix}
\end{dcases}
\]
For the longitudinal term we have:
\[
\grad \mathfrak L\of \gl = \pa{-m C_{n m} \sin m \gl + m S_{n m} \cos m \gl} \grad \gl
= \frac{m}{r \cos \gb} \pa{S_{n m} \cos m \gl - C_{n m} \sin m \gl} \begin{pmatrix}
-\sin \gl \\
\cos \gl \\
0
\end{pmatrix}
\]
\subsection*{Singularities}
While the above formulæ are all we need to compute the gradient of the geopotential, they present a number of singularities that require some care for implementation purposes.

There is obviously a pole when $r = 0$.  This one is not very interesting as we are never going to compute the acceleration at the centre of the celestial.

There are however removable singularities that arise when $\cos \gb$ appears at the denominator: any point on the axis of rotation of the celestial has $\cos \gb = 0$, but clearly the acceleration there is finite.

Consider $\grad \mathfrak B\of \gb$.  When $m = 0$ it includes a term in $\pa{\cos \gb}^{-1}$.  However, that term is multiplied by $m$, so it vanishes.  Thus, for $m = 0$ we must use the special formula:
\[
{\grad \mathfrak B\of \gb}_{m = 0} =  \frac{\cos \gb}{r}
\Evaluate{\deriv{t}{\LegendrePolynomial{n}\of t}}{t = \sin \gb}\qquad\begin{pmatrix}
-\sin \gb \cos \gl \\
-\sin \gb \sin \gl \\
\cos \gb
\end{pmatrix}
\]
Similarly $\grad \mathfrak L\of \gl$ has $\cos \gb$ as its denominator.  To eliminate this term we note that $\grad \mathfrak L\of \gl$ always occurs in a product involving $\mathfrak B\of \gb$.  Let's write this product:
\[
{\mathfrak B\of \gb}{\grad \mathfrak L\of \gl} = \frac{m \pa{\cos \gb}^{m - 1}}{r}
\Evaluate{\deriv[m]{t}{\LegendrePolynomial{n}\of t}}{t = \sin \gb}
\pa{S_{n m} \cos m \gl - C_{n m} \sin m \gl}\begin{pmatrix}
-\sin \gl \\
\cos \gl \\
0
\end{pmatrix}
\]
When $m = 0$ this expression has a term in $\pa{\cos \gb}^{-1}$ so we need to special-case it:
\[
{\mathfrak B\of \gb}{\grad \mathfrak L\of \gl}_{m = 0} = 0
\]
\subsection*{Implementation considerations}
A naïve implementation of the preceding formulæ yields poor performance because of the computation of trigonometric functions, powers and polynomials.  It is advantageous
to use simple geometric identities and recurrence formulæ to avoid these computations, at the expense of a small amount of storage.  First, we note that:
\[
x^2 + y^2 = r^2 \cos^2 \gb
\]
and therefore we obtain the trigonometric functions for $\gb$:
\[
\begin{dcases}
\cos \gb = \frac{\sqrt{x^2 + y^2}}{r} \\
\sin \gb = \frac{z}{r}
\end{dcases}
\]
and for $\gl$:
\[
\begin{dcases}
\cos \gl = \frac{x}{\sqrt{x^2 + y^2}} \\
\sin \gl = \frac{y}{\sqrt{x^2 + y^2}}
\end{dcases}
\]
In the case where $\sqrt{x^2 + y^2}$ is $0$, any value will do for $\gl$, so we can for instance pick $\cos \gl = 1$ and $\sin \gl = 0$.

The value of $\cos^m \gb$ may be computed by recurrence with a relative error growing like $\BigO\of m$.  If $m$ is even, we have:
\[
\begin{dcases}
h = \frac m 2 \\
\cos^m \gb = \pa{\cos^h \gb}^2
\end{dcases}
\]
and if $m$ is odd:
\[
\begin{dcases}
h_1 = \Floor{\frac m 2} \\
h_2 = m - h_1 \\
\cos^m \gb = \pa{\cos^{h_1} \gb}\pa{\cos^{h_2} \gb}
\end{dcases}
\]
The value of $\cos m \gl$ and $\sin m \gl$ may similarly be computed using elementary trigonometric identities with an absolute error growing like $\BigO\of {m^2}$.  If $m$ is even, we have:
\[
\begin{dcases}
h = \frac m 2 \\
\cos m \gl = \pa{\cos h \gl + \sin h \gl}\pa{\cos h \gl - \sin h \gl} \\
\sin m \gl = 2 \sin h \gl \cos h \gl
\end{dcases}
\]
and if $m$ is odd:
\[
\begin{dcases}
h_1 = \Floor{\frac m 2} \\
h_2 = m - h_1 \\
\sin m \gl = \sin h_1 \gl \cos h_2 \gl + \cos h_1 \gl \sin h_2 \gl \\
\cos m \gl = \cos h_1 \gl \cos h_2 \gl - \sin h_1 \gl \sin h_2 \gl
\end{dcases}
\]
For the Legendre polynomials, we are looking for a recurrence relationship among terms of the form $\deriv[m]{t}{\LegendrePolynomial{n}\of t}$.  First note that these terms are $0$ when $m > n$.  We start with the recurrence defining the polynomials themselves:
\[
n \LegendrePolynomial{n}\of t = \pa{2 n - 1} t \LegendrePolynomial{n - 1}\of t - \pa{n - 1} \LegendrePolynomial{n - 2}\of t
\]
and we derive $m$ times:
\[
n \deriv[m]{t}{\LegendrePolynomial{n}\of t} = \pa{2 n - 1} \deriv[m]{t}{\pa{t \LegendrePolynomial{n - 1}\of t}} - \pa{n - 1} \deriv[m]{t}{\LegendrePolynomial{n - 2}\of t}
\]
It's easy to verify, by recurrence, that:
\[
\deriv[m]{t}{\pa{t \LegendrePolynomial{n}\of t}} = t \deriv[m]{t}{\LegendrePolynomial{n}\of t} + m \deriv[m - 1]{t}{\LegendrePolynomial{n}\of t}
\]
so we end up with:
\[
n \deriv[m]{t}{\LegendrePolynomial{n}\of t} =
\pa{2 n - 1}\pa{t \deriv[m]{t}{\LegendrePolynomial{n - 1}\of t} + m \deriv[m - 1]{t}{\LegendrePolynomial{n - 1}\of t}} -
\pa{n - 1} \deriv[m]{t}{\LegendrePolynomial{n - 2}\of t}
\]
This recurrence starts with ${\LegendrePolynomial{0}\of t} = 1$, ${\LegendrePolynomial{1}\of t} = t$ and ${\deriv{t}{\LegendrePolynomial{1}\of t}} = 1$.  It is similar to \cite[eqn. (12)]{Westra2017} and makes it possible to efficiently evaluate $\Evaluate{\deriv[m]{t}{\LegendrePolynomial{n}\of t}}{t = \sin \gb}$ for increasing values of $n$ and $m$.

\section*{Damping}
In order to reduce the computational cost, we ignore high degree and order terms at long range.
This is done by replacing $\mathfrak R\of{r}$ by $\gs\of r \mathfrak R\of r$ in $V_{nm}$, where
$\gs\of r = 0$ for all $r>s_1$, and $\gs\of r = 1$ for all $r<s_0$, so that the $V_{nm}$ term
vanishes at long range and is unaffected at short range.

In the force computations, this means that $\grad\mathfrak R$ is replaced by\[
\grad \pa{\gs\mathfrak R} = \pa{\gs'\mathfrak R + \gs\mathfrak R'} \frac \vr r\text.
\]

\subsection*{Sigmoid}
Since we want the force to remain continuous, we need $\gs$ to be $\Cont[1]$; we achieve this by
making $\gs$ the cubic Hermite interpolation on the non-constant interval $\intclos {s_0} {s_1}$,
\[
\gs\of r=
\begin{cases}
1 &\text{for }r \leq s_0\text, \\
\frac{\pa{r-s_1}^2\pa{2r-3s_0+s_1}}{\pa{s_1-s_0}^3} &\text{for }r \in \intclos {s_0} {s_1}\text, \\
0 &\text{for }r \geq s_1\text.
\end{cases}
\]
The radial component of the acceleration resulting from $V_{nm}$ is 
\[
\gm\InnerProduct {\grad V_{nm}}{\frac \vr r} = \gm \deriv r {{\gs\of r \mathfrak R\of r}}{\mathfrak B\of \gb}\,{\mathfrak L\of \gl}\text.
\]
We require that it and all its derivatives be monotone, i.e., that
\[
\forall k \in \Positives, \deriv[k+1] r {\gs\of r \mathfrak R\of r}{\mathfrak B\of \gb}\,{\mathfrak L\of \gl}
\]
have constant sign for $r \in \intclos {s_0} {s_1}$.

Removing those factors that do not depend on $r$, this means that we must have
\begin{equation}
\forall r \in \intopen{s_0} {s_1}, \deriv[k+1] r {\gs\of r r^{-\pa{n+1}}} \neq 0\label{eqnderiv}\text.
\end{equation}

\begin{proposition}
The smallest value of $s_1$ such that (\ref{eqnderiv}) holds for all $k>0$ and $n≥2$ is $s_1 = 3s_0$.
\begin{proof}
We start by writing:
\[
\gs\of{r} = \sum{m = 0}[3] a_m r^m
\]
where:
\[
\begin{dcases}
a_0 &= \frac{\pa{s_1-3s_0}{s_1}^2}{\pa{s_1-s_0}^3} \\
a_1 &= \frac{6s_0 s_1}{\pa{s_1-s_0}^3} \\
a_2 &= \frac{-3\pa{s_0+s_1}}{\pa{s_1-s_0}^3} \\
a_3 &= \frac{2}{\pa{s_1-s_0}^3}
\end{dcases}
\]
The function being derived in (\ref{eqnderiv}) is:
\[
\gs\of{r} r^{-\pa{n+1}} = \sum{m = 0}[3] a_m r^{-\pa{n-m+1}}
\]
The p-th derivative of $r^{-q}$ for $q > 0$ is:
\[
\deriv[p] r {r^{-q}} = \pa{-1}^p \frac{\Factorial{\pa{q+p-1}}}{\Factorial{\pa{q-1}}} r^{-\pa{q+p}}
\]
Whence:
\begin{align*}
\deriv[k+1] r {\gs\of r r^{-\pa{n+1}}} &= \pa{-1}^{\pa{k+1}} \sum{m = 0}[3] a_m \frac{\Factorial{\pa{n+k-m+1}}}{\Factorial{\pa{n-m}}} r^{-\pa{n+k-m+2}} \\
&= \pa{-1}^{\pa{k+1}} r^{-\pa{n+k+2}} \frac{\Factorial{\pa{n+k-2}}}{\Factorial{n}}
\begin{aligned}[t]
&\;\Bigl[a_0 \pa{n+k+1}\pa{n+k}\pa{n+k-1} \\
&+a_1 \pa{n+k}\pa{n+k-1} n r \\
&+a_2 \pa{n+k-1} n \pa{n-1} r^2 \\
&+a_3 n \pa{n-1}\pa{n-2} r^3\Bigr]
\end{aligned}
\end{align*}
Substituting the values of the $\pa{a_i}$, we see that the zeroes of $\deriv[k+1] r {\gs\of r r^{-\pa{n+1}}}$ are those of the polynomial:
\begin{align*}
d_{n k}\of{s_0, s_1; r} &= \pa{s_1-3s_0}{s_1}^2 \pa{n+k+1}\pa{n+k}\pa{n+k-1} \\
&\quad + 6s_0 s_1 \pa{n+k}\pa{n+k-1} n r \\
&\quad -3\pa{s_0+s_1}\pa{n+k-1} n \pa{n-1} r^2 \\
&\quad + 2 n \pa{n-1}\pa{n-2} r^3
\end{align*}
The following three lemmata establish properties of the roots of $d_{n k}\of{s_0, s_1; r}$:
\begin{lemma}[necessary condition, general case]
If $\frac{9}{7}s_0<s_1<3s_0$ then there exist $k$ and $n$ such that $d_{n k}\of{s_0, s_1; r}$ has a zero in $\intopen{s_0} {s_1}$.
\end{lemma}
\begin{lemma}[necessary condition, special case]
If $s_0<s_1<\pa{1+\frac{\sqrt{2}}{2}}s_0$ then there exist $k$ and $n$ such that $d_{n k}\of{s_0, s_1; r}$ has a zero in $\intopen{s_0} {s_1}$.
\end{lemma}
\begin{lemma}[sufficient condition]
If $s_1 = 3s_0$ then the positive roots of $d_{n k}\of{s_0, s_1; r}$ are greater than $s_1$.
\end{lemma}
These lemmata trivially constitute necessary and sufficient conditions, respectively, for the proposition.
\end{proof}
\end{proposition}
\begin{proof}[of necessary condition, general case]
In this proof we set $n=2$, which causes the term in $r^3$ of $d_{n k}\of{s_0, s_1; r}$ to vanish:
\begin{multline*}
d_{2,k}\of{s_0, s_1; r} = \pa{s_1-3s_0}s_1^2\pa{k+3}\pa{k+2}\pa{k+1} \\+ 12s_0 s_1\pa{k+2}\pa{k+1}r - 6\pa{s_0+s_1}\pa{k+1}r^2
\end{multline*}
The roots of $d_{2,k}\of{s_0, s_1; r}$ can be computed by solving:
\[
\pa{s_1-3s_0}s_1^2\pa{k+3}\pa{k+2} + 12s_0 s_1\pa{k+2}r - 6\pa{s_0+s_1}r^2 = 0
\]
The reduced discriminant of this equation is:
\begin{align*}
\gD' &= 36 s_0^2 s_1^2 \pa{k+2}^2 + 6\pa{s_0+s_1}\pa{s_1-3s_0}s_1^2\pa{k+2}\pa{k+3}\\
&= 6 s_1^2\pa{k+2}\pa{6 s_0^2\pa{k+2}+\pa{s_1^2-2 s_0 s_1-3s_0^2}\pa{k+3}}\\
&= 6 s_1^2\pa{k+2}\pa{k\pa{s_1^2-2 s_0 s_1 + 3s_0^2} + 3\pa{s_1-s_0}^2}
\end{align*}
and its solutions are:
\[
r_± = s_1\frac{6 s_0\pa{k+2} ± \sqrt{6\pa{k+2}\pa{k\pa{s_1^2-2 s_0 s_1 + 3s_0^2} + 3\pa{s_1-s_0}^2}}}{6\pa{s_0+s_1}}
\]
We will now define $k_1$ and $k_2$ such that the following claim holds:
\begin{claim}
For every $k \in \intopen{k_2}{k_1}\Intersection\intclop{2}{+\infty}\Intersection{\Naturals}$, $r_- \in \intopen{s_0}{s_1}$.
\begin{proof}
First, consider the inequality $r_-<s_1$.  It can be rewritten as follows:
\begin{align*}
&6 s_0\pa{k+2} - 6\pa{s_0+s_1} < \sqrt{6\pa{k+2}\pa{k\pa{s_1^2-2 s_0 s_1 + 3s_0^2} + 3\pa{s_1-s_0}^2}}
\end{align*}
If $k≥2$ the left-hand side is positive and we can further rewrite the inequality:
\begin{align*}
&6\pa{k s_0 + \pa{s_0-s_1}}^2 < \pa{k+2}\pa{k\pa{s_1^2-2 s_0 s_1 + 3s_0^2} + 3\pa{s_1-s_0}^2} \\
\iff &6\pa{k^2 s_0^2 + 2 k s_0 \pa{s_0 - s_1} + \pa{s_0 - s_1}^2} < \\
&\qquad k^2 \pa{s_1^2-2 s_0 s_1 + 3s_0^2} + k \pa{3\pa{s_1-s_0}^2 + 2\pa{s_1^2-2 s_0 s_1 + 3s_0^2}} + 6\pa{s_0 - s_1}^2 \\
\iff &k^2\pa{3s_0^2 + 2 s_0 s_1 - s_1^2} + k\pa{12 s_0\pa{s_0-s_1} - 3\pa{s_1-s_0}^2 - 2\pa{s_1^2-2 s_0 s_1 + 3s_0^2}} < 0 \\
\iff &k\pa{3s_0^2 + 2 s_0 s_1 - s_1^2} + \pa{3s_0^2 -2 s_0 s_1 - 5 s_1^2} < 0 \\
\iff &k\pa{3s_0-s_1}\pa{s_0+s_1} + \pa{3s_0 - 5s_1}\pa{s_0+s_1} < 0 \\
\iff &k < \frac{5s_1-3s_0}{3s_0-s_1}=:k_1
\end{align*}
Similarly, the inequality $s_0<r_-$ can be rewritten as follows:
\begin{align*}
&6 s_0 s_1\pa{k+2} - 6\pa{s_0+s_1}s_0 > s_1 \sqrt{6\pa{k+2}\pa{k\pa{s_1^2-2 s_0 s_1 + 3s_0^2} + 3\pa{s_1-s_0}^2}} \\
\iff &6 s_0^2\pa{k s_1 + \pa{s_1-s_0}}^2 > s_1^2\pa{k+2}\pa{k\pa{s_1^2-2 s_0 s_1 + 3s_0^2} + 3\pa{s_1-s_0}^2} \\
\iff &6 s_0^2\pa{k^2 s_1^2 + 2 k s_1\pa{s_1-s_0} + \pa{s_1-s_0}^2} > \\
&\qquad k^2 s_1^2\pa{s_1^2-2 s_0 s_1 + 3s_0^2} + k s_1^2\pa{3\pa{s_1-s_0}^2 + 2\pa{s_1^2-2 s_0 s_1 + 3s_0^2}} + 6 s_1^2\pa{s_0 - s_1}^2 \\
\iff &k^2 s_1^2\pa{3s_0^2 + 2 s_0 s_1 - s_1^2} + k s_1\pa{-5s_1^3 + 10 s_0 s_1^2 + 3 s_0^2 s_1 - 12 s_0^3} + 6\pa{s_0-s_1}^3\pa{s_0+s_1} > 0 \\
\iff &k^2 s_1^2\pa{3s_0-s_1}\pa{s_0+s_1} + k s_1\pa{-5s_1^2 + 15 s_0 s_1 - 12s_0^2}\pa{s_0+s_1} + 6\pa{s_0-s_1}^3\pa{s_0+s_1} > 0 \\
\iff &k^2 s_1^2\pa{3s_0-s_1} + k s_1\pa{-5s_1^2 + 15 s_0 s_1 - 12s_0^2} + 6\pa{s_0-s_1}^3 > 0
\end{align*}
When $s_1<3s_0$, the coefficient of $k^2$ in the polynomial above is positive.  The constant term is negative, therefore the product of the roots is negative.  For the inequality to be true, $k$ must be greater than the largest root the polynomial.  The discriminant of the polynomial is:
\[
\gD = s_1^2\pa{\pa{-5s_1^2 + 15 s_0 s_1 - 12s_0^2}^2 - 24\pa{3s_0-s_1}\pa{s_0-s_1}^3}
\]
And the largest root is:
\[
k_2 = \frac{-s_1\pa{-5s_1^2 + 15 s_0 s_1 - 12s_0^2} + \sqrt{\gD}}{2s_1^2\pa{3s_0-s_1}}
\]
\end{proof}
\end{claim}
\begin{claim}
$\intopen{k_2}{k_1}\Intersection\intclop{2}{+\infty}\Intersection{\Naturals}$ is nonempty when $s_1>\frac{9}{7}s_0 \approx 1.286s_0$.
\begin{proof}
For this set to be nonempty we must have $k_1>2$ which is equivalent to $s_1>\frac{9}{7}s_0$.  Given $k_1>2$, a sufficient condition for the interval $\intopen{k_2}{k_1}$ to contain an integer greater than $1$ is $k_1-k_2>1$.  This inequality can be rewritten as follows:
\begin{align*}
&2s_1^2\pa{5s_1-3s_0} + s_1\pa{-5s_1^2 + 15 s_0 s_1 - 12s_0^2} - 2s_1^2\pa{3s_0-s_1} - \sqrt{\gD} > 0 \\
\iff &s_1\pa{7s_1^2 + 3 s_0 s_1 - 12s_0^2} > \sqrt{\gD}
\end{align*}
This inequality can only be true if the left-hand side is positive, that is, if $s_1 ≥ \frac{\sqrt{345}-3}{14}s_0 \approx 1.112s_0$.  Assuming that this condition is met we can further rewrite the inequality:
\begin{align*}
&\pa{7s_1^2 + 3 s_0 s_1 - 12s_0^2}^2 > \pa{-5s_1^2 + 15 s_0 s_1 - 12s_0^2}^2 - 24\pa{3s_0-s_1}\pa{s_0-s_1}^3 \\
\iff &49s_1^4 + 42 s_0 s_1^3 - 159 s_0^2 s_1^2 - 72 s_0^3 s_1 + 144 s_0^4 > \\
&\qquad s_1^4 - 6 s_0 s_1^3 + 57 s_0^2 s_1^2 - 120 s_0^3 s_1 + 72 s_0^4 \\
\iff &24\pa{s_1-s_0}\pa{2s_1^3 + 4 s_0 s_1^2 - 5 s_0^2 s_1 - 3s_0^3} > 0
\end{align*}
The sum of the roots of $2s_1^3 + 4 s_0 s_1^2 - 5 s_0^2 s_1 - 3s_0^3$ is $-2s_0$ and the product is $3s_0/2$.  This means that this polynomial has a single positive root in $s_1$ and that when $s_1$ is greater than this root the inequality above holds.  Solving the cubic using $\emph{Mathematica}$ yields this root:
\begin{align*}
\hat\gq &= \frac{1}{3} \arctan\of{\frac{3 \sqrt{2517}}{41}} \\
\hat{s}_1 &= \frac{s_0}{6}\pa{-4+\sqrt{138}\sin\hat\gq + \sqrt{46}\cos\hat\gq} \approx 1.183s_0
\end{align*}
Noting that $\hat{s}_1<\frac{9}{7}s_0$, we conclude that, if $s_1 \in \intopen{\frac{9}{7}s_0}{3s_0}$ it is possible to find an integer $k≥2$ such that the smallest root of $d_{2,k}\of{s_0, s_1; r}$ is in $\intopen{s_0} {s_1}$.
\end{proof}
\end{claim}
\end{proof}

\begin{proof}[of necessary condition, special case]
In this proof we set $n=2$ and $k=1$.
\begin{claim}
If $s_1<2s_0$, then for every $k \in \intopen{k_2}{k_1}\Intersection{\Positives}$, $r_- \in \intopen{s_0}{s_1}$.
\begin{proof}
The inequality $r_-<s_1$ can be rewritten as in the previous proof, with the caveat that since the condition $k≥2$ does not hold anymore, we have to require $s_1<2s_0$ for the quantity $s_0\pa{k+2}-\pa{s_0+s_1}$ to be positive.  This is acceptable since we are looking for a proof when $s_1$ is close to $s_0$.  The inequality $s_0<r_-$ is unaffected by any condition on $k$ so the definition of $k_2$ remains valid.
\end{proof}
\end{claim}
\begin{claim}
$1 \in \intopen{k_2}{k_1}$ when $s_1<\pa{1+\frac{\sqrt{2}}{2}}s_0 \approx 1.707s_0$.
\begin{proof}
The inequality $1<k_1$ can be rewritten:
\[
1 < \frac{5s_1-3s_0}{3s_0-s_1} \iff s_0 > s_1 \qquad\text{which is trivially true.}
\]
The inequality $s_0<r_-$ can be rewritten as in the previous proof and becomes, substituting $k$:
\begin{align*}
&k^2 s_1^2\pa{3s_0-s_1} + k s_1\pa{-5s_1^2 + 15 s_0 s_1 - 12s_0^2} + 6\pa{s_0-s_1}^3 > 0 \\
\iff &s_1\pa{-6s_1^2 + 18 s_0 s_1 - 12 s_0^2} + 6\pa{s_0-s_1}^3 > 0 \\
\iff &-2s_1^3 + 6 s_0 s_1^2 - 5 s_0^2 s_1 + s_0^3 > 0 \\
\iff &\pa{s_1 - s_0}\pa{-2 s_1^2 + 4 s_0 s_1 - s_0^2} > 0
\end{align*}
The quadratic term in the above condition is positive between its roots, and its only root greater than $s_0$ is $\pa{1+\frac{\sqrt{2}}{2}}s_0$.  Thus, if $s_1 \in \intopen{s_0}{\pa{1+\frac{\sqrt{2}}{2}}s_0}$ the smallest root of $d_{2,1}\of{s_0, s_1; r}$ is in $\intopen{s_0} {s_1}$.
\end{proof}
\end{claim}
\end{proof}

\begin{proof}[of sufficient condition]
With the choice $s_1=3s_0$ the constant term of the polynomial $d_{n k}\of{s_0, s_1; r}$ vanishes and $r=0$ is a trivial root.  We are left with the equation:
\[
6s_0 s_1 \pa{n+k}\pa{n+k-1} - 3\pa{s_0+s_1}\pa{n+k-1} \pa{n-1} r + 2 \pa{n-1}\pa{n-2} r^2 = 0
\]
or, substituting $s_1$:
\begin{equation}
9{s_0}^2 \pa{n+k}\pa{n+k-1} - 6 s_0 \pa{n+k-1} \pa{n-1} r + \pa{n-1}\pa{n-2} r^2 = 0 \label{eqn3s0}
\end{equation}
The reduced discriminant of equation (\ref{eqn3s0}) is:
\begin{align*}
\gD' &= 9 {s_0}^2 \pa{n+k-1}^2 \pa{n-1}^2 - 9 {s_0}^2 \pa{n+k}\pa{n+k-1}\pa{n-1}\pa{n-2}\\
&=9 {s_0}^2\pa{n+k-1}\pa{n-1}\pa{k+1}
\end{align*}
and its solutions are, when $n>2$:
\[
r_± = 3 s_0\frac{\pa{n+k-1}\pa{n-1} ± \sqrt{\pa{n+k-1}\pa{n-1}\pa{k+1}}}{\pa{n-1}\pa{n-2}}
\]
We want to prove that the smallest root, $r_-$ verifies $r_{-}>3s_0$.  This inequality can be rewritten:
\begin{align*}
&\pa{n+k-1}\pa{n-1} - \pa{n-1}\pa{n-2} > \sqrt{\pa{n+k-1}\pa{n-1}\pa{k+1}} \\
\iff &\pa{n-1}\pa{k+1} > \sqrt{\pa{n+k-1}\pa{n-1}\pa{k+1}} \\
\iff &\pa{n-1}\pa{k+1} > n+k-1 \\
\iff &k\pa{n-2} > 0 \quad \text{which is trivally true.}
\end{align*}
When $n=2$ equation (\ref{eqn3s0}) further reduces to a first-degree equation with root $r = \frac{3}{2} s_0 \pa{k+2}$ which verifies $r≥3s_0$.  Note however that the latter inequality is not strict, i.e., the value $3s_0$ may be reached if $k=0$ and $n=2$.
\end{proof}
Accordingly, we pick $s_1=3s_0$.
The expression for $\gs$ on $\intclos {s_0} {s_1}$ simplifies to\[
\gs\of r = \frac{r \pa{r-3s_0}^2} {4 s_0^3} = \frac{r^3}{4 s_0^3} - \frac{3 r^2}{2s_0^2} + \frac{9r}{4s_0}\text.
\]

\subsection*{Threshold}
For a tolerance $\gce$, we choose $s_0$ as the smallest distance such that the radial component of the force at $s_0$
does not exceed $\gce$ times the magnitude of the central force,
\[
\max_{\gl,\gb} \abs{s_0^2\mathfrak R' \of {s_0} \mathfrak B \of \gb \mathfrak L\of\gl}
= \gce\text.
\]
This maximum decomposes to
\[
\max_{\gl,\gb} \abs{s_0^2\mathfrak R' \of {s_0} \mathfrak B \of \gb \mathfrak L\of\gl}
= \abs{s_0^2\mathfrak R' \of {s_0}} \max_\gb\abs{\mathfrak B \of \gb} \max_\gl \abs{\mathfrak L\of\gl}\text.
\]
The first factor is
\[
\abs{s_0^2\mathfrak R' \of {s_0}}=s_0^2\frac{\pa{n+1}\mathfrak R\of{s_0}}{s_0}=\pa{n+1}\frac{R^n}{s_0^n}\text.
\]
We have \begin{align*}
\max_\gl \abs{\mathfrak L\of\gl}
&= \max_\gl \abs{C_{nm}\cos m\gl + S_{nm} \sin m\gl}\\
&= \max_\gl \abs{\operatorname{Re}\of{\pa{C_{nm} + \I S_{nm}}\E^{-\I m\gl}}}\\
&= \abs{C_{nm} + \I S_{nm}} = \sqrt{C_{nm}^2 + S_{nm}^2}\text,
\end{align*}
and we numerically compute and tabulate \[
\max_\gb\abs{\mathfrak B \of \gb} = \max_\gb\abs{\p nm\of{\sin \gb}}\text.
\]
Thus we can compute the desired $s_0$ as
\[
s_0 = R \pa{\frac{\max_\gb\abs{\p nm\of{\sin \gb}}\pa{n+1}\sqrt{C_{nm}^2 + S_{nm}^2}}{\gce}}^{\frac 1 n}\text.
\]

\printbibliography[segment=0]
\newrefsegment
\nocite{*}
\end{document}