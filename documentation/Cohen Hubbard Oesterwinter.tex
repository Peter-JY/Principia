\documentclass[10pt, a4paper, twoside]{basestyle}

\usepackage[Mathematics]{semtex}
\usepackage{chngcntr}
\counterwithout{equation}{section}

%%%% Shorthands.

%%%% Title and authors.

\title{On a Formula by Cohen, Hubbard and Oesterwinter}
\date{\printdate{2021-03-13}}
\author{Robin~Leroy (eggrobin) \& Pascal~Leroy (phl)}
\begin{document}
\maketitle
\begin{sloppypar}
\noindent
This document proves and generalizes a formula given in \cite{CohenHubbardOesterwinter1973} to compute the velocity of a body in the context of numerical integration of $n$-body systems.
\end{sloppypar}

\subsection*{Statement}
\cite[20]{CohenHubbardOesterwinter1973} states that
\begin{quotation}
the formula for a velocity component [is] of the form:
\begin{equation}
\TimeDerivative{x}_{n + 1} = \frac{1}{h}\pa{x_n - x_{n - 1} + h^2 \sum{i = 0}[12] \gb_i \SecondTimeDerivative{x}_{n-i}}
\label{eqncho}
\end{equation}
\end{quotation}
They then proceed to tabulate explicit values for the (rational) coefficients $\gb_i$ without explaining how they are computed.  This makes it impossible to use this formula for integrators of a different order or to construct similar formulæ for slightly different purposes.

\subsection*{Finite difference formulæ}
In this section we prove a backward difference formula that makes it possible to compute an approximation of any derivative of a function on an equally-spaced grid at any desired order.  We then derive a corollary based on the second derivative $f\dersecond\of{x}$ that is useful for the following sections. 
\begin{lemma}
Given two integers $m ≤ n$, there exists a family of rational numbers $\gl^{m}_{n,\gn}$, with $0 ≤ \gn ≤ n$, such that for any sufficiently regular function $f$:

\begin{equation}
f^{(m)}\of{x_0} = \frac{\pa{-1}^m}{h^m} \sum{\gn = 0}[n]{\gl^{m}_{n,\gn} f\of{x_0 - \gn h}} + \BigO\of{h^{n - m + 1}} 
\label{fornbergbackward}
\end{equation}
\begin{proof}
\cite{Fornberg1988} gives finite difference formulæ for any order and for kernels of arbitrary size.  Given a family $\VectorSymbol{\ga} = \pa{\ga_0, \ga_1, ..., \ga_N}$ of points (which may not be equidistant), the $m$-th derivative at any point $x_0$ may be approximated to order $n - m + 1$ as:
\[
f^{(m)}\of{x_0} ≅ \sum{\gn = 0}[n]{\gd^{m}_{n,\gn}\of{\VectorSymbol{\ga}} f\of{\ga_\gn}}
\]
where $m ≤ n$ and the coefficients $\gd^{m}_{n,\gn}$ are dependent on $\VectorSymbol{\ga}$ but independent of $f$.

If the $\VectorSymbol{\ga}$ are equally spaced with step $-h$ (\idest, $\ga_\gn = x_0 - \gn h$) then we can write a backward formula as follows:
\begin{equation}
f^{(m)}\of{x_0} = \sum{\gn = 0}[n]{\gd^{m}_{n,\gn}\of{\VectorSymbol{\ga}} f\of{x_0 - \gn h}} + \BigO\of{h^{n - m + 1}}
\label{fornbergbackwardpreliminary}
\end{equation}
In this case equation (3.8) from \cite{Fornberg1988} may be rewritten as follows, restoring $x_0$:
\begin{align}
\gd^{m}_{n,\gn} &= \frac{1}{\ga_n - \ga_\gn}\pa{\pa{\ga_n - x_0} \gd^{m}_{n - 1,\gn} - m \gd^{m - 1}_{n - 1,\gn}} \nonumber \\
&= \frac{1}{\pa{\gn - n} h} \pa{-n h \gd^{m}_{n - 1,\gn} - m \gd^{m - 1}_{n - 1,\gn}} \label{eqndelta}
\end{align}
It is possible to extract the powers of $h$ from the coefficients by defining, for $\gn < n$:
\[
\gl^{m}_{n,\gn} \DefineAs \gd^{m}_{n,\gn}\of{\VectorSymbol{\ga}} \pa{-1}^m h^m
\]
Substituting into equation (\ref{eqndelta}) we obtain:
\[
\gl^{m}_{n,\gn} = \frac{1}{n - \gn} \pa{n \gl^{m}_{n - 1,\gn} - m \gl^{m - 1}_{n - 1,\gn}}
\]
and we see that $\gl^{m}_{n,\gn}$ is independent from $x_0$ and $h$ and therefore that $\gd^{m}_{n,\gn}$ is independent from $x_0$.
Note that the definition of $\gl^{m}_{n,\gn}$ extends immediately to $n = \gn$, that is to $\gl^{m}_{n,n}$, using equation (3.10) of \cite{Fornberg1988}.

Substituting:
\[
\gd^{m}_{n,\gn}\of{\VectorSymbol{\ga}} = \pa{-1}^m \frac{\gl^{m}_{n,\gn}}{h^m}
\]
into equation (\ref{fornbergbackwardpreliminary}) yields equation (\ref{fornbergbackward}), thus proving the lemma.

% https://tex.stackexchange.com/questions/79194/misplaced-qed-symbol-after-displaymath-inside-item-of-inline-list
\vspace{-\belowdisplayskip}\[\]
\end{proof}
\end{lemma}

\begin{corollary}
Applying the lemma to the second derivative $f\dersecond\of{x}$ gives the following result when $m ≥ 2$:
\begin{equation}
f^{(m)}\of{x_0} = \frac{\pa{-1}^m}{h^{m - 2}} \sum{\gn = 0}[n - 2]{\gl^{m - 2}_{n - 2,\gn} f\dersecond\of{x_0 - \gn h}} + \BigO\of{h^{n - m + 1}}
\label{eqnfornbergbackward2}
\end{equation}
\end{corollary}

\subsection*{A formula for symmetric linear multistep integrators}
In this section we derive a formula suitable for computing the velocity of a body knowing its positions and accelerations at preceding times.  This is the formula we use after integration using a symmetry linear multistep integrator.

\begin{proposition}
There exists a family of rational numbers $\gh_{n, \gn}$, with $0 ≤ \gn ≤ n - 2$, such that for any sufficiently regular function $f$: 
\begin{equation}
f\der\of{x_0} = \frac{1}{h} \pa{f\of{x_0} - f\of{x_0 - h} + h^2 \sum{\gn = 0}[n - 2]{\gh_{n, \gn} f\dersecond\of{x_0 - \gn h}}} + \BigO\of{h^n} 
\label{eqnslms}
\end{equation}
\begin{proof}
We start by writing the Taylor series of $f\of{x_0 - h}$ for $h$ close to $0$, extracting the leading terms:
\begin{align}
f\of{x_0 - h} &= \sum{m = 0}[\infty]{\frac{f^{(m)}\of{x_0}}{\Factorial{m}} \pa{-1}^m h^m} \nonumber \\
&=f\of{x_0} - h f\der\of{x_0} + \sum{m = 2}[\infty]{\frac{f^{(m)}\of{x_0}}{\Factorial{m}} \pa{-1}^m h^m} \nonumber \\
&=f\of{x_0} - h f\der\of{x_0} + \sum{m = 2}[n]{\frac{f^{(m)}\of{x_0}}{\Factorial{m}} \pa{-1}^m h^m} + \BigO\of{h^{n+1}} \label{eqntaylorn}
\end{align}
Injecting the expression from corollary (\ref{eqnfornbergbackward2}) into this series we find:
\[
f\of{x_0 - h} = f\of{x_0} - h f\der\of{x_0} + h^2 \sum{m = 2}[n]{~\sum{\gn = 0}[n - 2]{\frac{\gl^{m - 2}_{n - 2,\gn}}{\Factorial m} f\dersecond\of{x_0 - \gn h}}} + \BigO\of{h^{n + 1}}
\]
The summations are independent and can be exchanged:
\[
f\of{x_0 - h} = f\of{x_0} - h f\der\of{x_0} + h^2 \sum{\gn = 0}[n - 2]{\pa{\sum{m = 2}[n]{\frac{\gl^{m - 2}_{n - 2,\gn}}{\Factorial m}}} f\dersecond\of{x_0 - \gn h}} + \BigO\of{h^{n + 1}} \\
\]
If we define:
\[
\gh_{n, \gn} \DefineAs \sum{m = 2}[n]{\frac{\gl^{m - 2}_{n - 2,\gn}}{\Factorial m}}
\]
the equation (\ref{eqnslms}) follows immediately.

\vspace{-\belowdisplayskip}\[\]
\end{proof}
\end{proposition}

\subsection*{The Cohen-Hubbard-Osterwinder formula}
In this section we derive the Cohen-Hubbard-Osterwinder formula, equation (\ref{eqncho}).

\begin{proposition}
There exists a family of rational numbers $\gb_{n, \gn}$, with $0 ≤ \gn ≤ n - 2$, such that for any sufficiently regular function $f$:
\[
f\der\of{x_0 + h} = \frac{1}{h} \pa{f\of{x_0} - f\of{x_0 - h}} + h \sum{\gn = 0}[n - 2]{\gb_{n, \gn} f\dersecond\of{x_0 - \gn h}} + \BigO\of{h^n}
\]
\begin{proof}
We start by writing the Taylor series of $f\der\of{x_0 + h}$ for $h$ close to $0$:
\begin{align*}
f\der\of{x_0 + h} &= f\der\of{x_0} + \sum{m = 2}[\infty]{\frac{h^{m - 1}}{\Factorial{\pa{m - 1}}} f^{(m)}\of{x_0}} \\
&= f\der\of{x_0} + \sum{m = 2}[n]{\frac{h^{m - 1}}{\Factorial{\pa{m - 1}}} f^{(m)}\of{x_0}} + \BigO\of{h^n}
\end{align*}
and we replace $f\der\of{x_0}$ with its value from equation (\ref{eqntaylorn}):
\[
f\der\of{x_0 + h} = \frac{1}{h} \pa{f\of{x_0} - f\of{x_0 - h}} + \sum{m = 2}[n]{\frac{h^{m - 1}}{\Factorial{\pa{m - 1}}} \pa{1 + \frac{\pa{-1}^m}{m}} f^{(m)}\of{x_0}} + \BigO\of{h^n}
\]

As we did above, we replace $f^{(m)}\of{x_0}$ using its expression from corollary (\ref{eqnfornbergbackward2}) to obtain:
\[
f\der\of{x_0 + h} = \frac{1}{h} \pa{f\of{x_0} - f\of{x_0 - h}} + h \sum{m = 2}[n]{
\pa{\frac{1}{\Factorial{\pa{m - 1}}} \pa{\pa{-1}^m + \frac{1}{m}} \sum{\gn = 0}[n - 2]{\gl^{m - 2}_{n - 2, \gn}} f\dersecond\of{x_0 - \gn h}}
} + \BigO\of{h^n}
\]
If we exchange the summations and define:
\[
\gb_{n, \gn} = \sum{m = 2}[n]{\frac{\gl^{m - 2}_{n - 2, \gn}}{\Factorial{\pa{m - 1}}} \pa{\pa{-1}^m + \frac{1}{m}}}
\]
we obtain the desired result, which is a reformulation of equation (\ref{eqncho}) where $\gb_i$ in \cite{CohenHubbardOesterwinter1973} is our $\gb_{14, i}$.
\end{proof}
\end{proposition}

\end{document}