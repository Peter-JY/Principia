\documentclass[10pt, a4paper, twoside]{basestyle}
\usepackage{tkz-euclide}
\usepackage[Mathematics]{semtex}
\usepackage{chngcntr}
\counterwithout{equation}{section}

%%%% Shorthands.

%%%% Title and authors.

\newcommand{\point}[1]{\mathrm{#1}}
\newcommand{\bipoint}[2]{\overrightarrow{\point #1 \point #2}}
\newcommand{\straightline}[2]{\point #1 \point #2}
\newcommand{\plane}[3]{\point #1 \point #2 \point #3}
\newcommand{\squarenorm}[1]{\scal{#1}{#1}}

\title{Hiding Computations in Projection}
\date{\printdate{2017-08-17}}
\author{Pascal~Leroy (pleroy)}
\begin{document}
\maketitle
This document describes the computations that are performed by the method
\texttt{VisibleSegments} of class \texttt{Perspective} to determine the parts of
a segment that are hidden by a sphere when seen from a pinhole camera.

Figure~\ref{fig3d} illustrates an example of the problem.  The points drawn in
black define the problem: $\point A$ and $\point B$ are the extremities of the
segment; $\point C$ is the centre of the sphere; $R$ is the radius of the
sphere; and $\point K$ is the location of the camera.  The points drawn in red
will be computed as part of the resolution.  Of particular interest is the plane
$\plane KAB$, in which we willl do much of the analysis below.  The figure shows
the circle formed by the intersection of this plane with the sphere, as well as
$\point H$, the center of that circle and the projection of $\point C$ on
$\plane KAB$.
\begin{figure}[htb!]
\centering
\includegraphics[scale=0.35]{Perspective-hiding-3d}
\caption{A 3-dimensional example.  In this case, the only segment visible from
$\point K$ is the part of $\straightline AB$ between $\point B$ and the red
point immediately to its left.\label{fig3d}}
\end{figure}

\subsection*{Camera inside the sphere}
We start our analysis by eliminating a case that would cause anomalies in the
analysis below.  If\[
\scal{\bipoint KC}{\bipoint KC} < R^2
\]
then the camera is inside the sphere and the segment is hidden irrespective of
its position.

\subsection*{Sphere and segment in distinct half-spaces}
Consider the plane containing $\point K$ and orthogonal to $\straightline KC$;
it separates the entire space into two half-spaces.  If the segment
$\straightline AB$ is entirely within the half-space that does not contain
$\point C$ then the segment is not hidden (remember that $\point K$ is not
inside the sphere).  This is the case if the following inequalities are both
true:
\begin{align*}
\scal{\bipoint KA}{\bipoint KC} &< 0\\
\scal{\bipoint KB}{\bipoint KC} &< 0\text.
\end{align*}

\subsection*{Projection of $\point C$ on $\plane KAB$}
For simplicity we will do the rest of our analysis in the plane $\plane KAB$ and
we will often use $\tuple{\bipoint KA, \bipoint KB}$ as a basis of that plane.
Let $\point H$ be the orthogonal projection of $\point C$ on $\plane KAB$.
and define $\ga$, $\gb$ to be its coordinates in
$\tuple{\bipoint KA, \bipoint KB}$
\[
\bipoint KH = \ga \bipoint KA + \gb \bipoint KB\text.
\]
Note that $\bipoint KH = \bipoint KC + \bipoint CH$. By definition,
$\bipoint CH$ is orthogonal to both $\bipoint KA$ and $\bipoint KB$:
\begin{align*}
\scal{\bipoint KA}{\bipoint CH} &= 0\\
\scal{\bipoint KB}{\bipoint CH} &= 0\text.
\end{align*}
Decomposing $\bipoint CH$ we obtain:
\begin{align*}
\scal{\bipoint KA}{\bipoint KH} &= \scal{\bipoint KA}{\bipoint KC}\\
\scal{\bipoint KB}{\bipoint KH} &= \scal{\bipoint KB}{\bipoint KC}\text.
\end{align*}
Expanding $\bipoint KH$ on the basis $\tuple{\bipoint KA, \bipoint KB}$ gives a
linear system of two equations with two unknowns:
\begin{align*}
\ga \scal{\bipoint KA}{\bipoint KA} + \gb \scal{\bipoint KA}{\bipoint KB}
    &= \scal{\bipoint KA}{\bipoint KC}\\
\ga \scal{\bipoint KA}{\bipoint KB} + \ga \scal{\bipoint KB}{\bipoint KB}
    &= \scal{\bipoint KB}{\bipoint KC}\text.
\end{align*}
The determinant of this system is\[
D = \pascal{\squarenorm{\bipoint KA}} \pascal{\squarenorm{\bipoint KB}} -
    \pascal{\scal{\bipoint KA}{\bipoint KB}}^2\text,
\]
which is non-zero if and only if $\point A \neq \point B$. The solutions are
thus:
\begin{align*}
\ga &= \frac
  {\pascal{\squarenorm{\bipoint KB}} \pascal{\scal{\bipoint KA}{\bipoint KC}} -
   \pascal{\scal{\bipoint KA}{\bipoint KB}}
   \pascal{\scal{\bipoint KB}{\bipoint KC}}}
  {D}\\
\gb &= \frac
  {\pascal{\squarenorm{\bipoint KA}} \pascal{\scal{\bipoint KB}{\bipoint KC}} -
   \pascal{\scal{\bipoint KA}{\bipoint KB}}
   \pascal{\scal{\bipoint KA}{\bipoint KC}}}
  {D}\text.
\end{align*}
Once $\bipoint KH$ is determined we can compute $\bipoint CH =
\bipoint KH - \bipoint KC$.  If $\squarenorm{\bipoint CH} \geq R^2$, the sphere
is either tangent to the plane $\plane KAB$ or doesn't intersect it. In these
cases, there is no hiding.

If the sphere intersects $\plane KAB$, then the intersection is a circle whose
radius we denote by $r$ in the rest of this analysis.  The radius is such that
$r^2 = R^2 - \squarenorm{\bipoint CH}$.

\subsection*{An optimization where the circle is outside the wedge $\plane KAB$}
\marginfig[Circle lying outside the wedge $\plane KAB$ on the side of
$\straightline KB$.\label{figOutside}]{
\begin{tikzpicture}[scale=0.5]
\tkzInit[ymin=-7,ymax=8,xmin=-1,xmax=9]
\tkzClip
\tkzDefPoint(3,5){A}
\tkzDefPoint(8,0){B}
\tkzDefPoint(4,-3){H}
\tkzDefPoint(0,0){K}

\tkzDefLine[orthogonal=through H](K,B) \tkzGetPoint{h1}
\tkzInterLL(H,h1)(K,B) \tkzGetPoint{N}

\tkzDefLine[parallel=through H](K,A) \tkzGetPoint{h2}
\tkzInterLL(H,h2)(K,B) \tkzGetPoint{M}

\tkzDefPointWith[linear,K=2](H,M) \tkzGetPoint{m}

\tkzDefPointBy[rotation=center H angle -130](N) \tkzGetPoint{r}

\tkzDrawPoints(A,B,H,K,M,N)
\tkzLabelPoint[left](A){$\point A$}
\tkzLabelPoint[above](B){$\point B$}
\tkzLabelPoint[left](H){$\point H$}
\tkzLabelPoint[above left](K){$\point K$}
\tkzLabelPoint[below right](M){$\point M$}
\tkzLabelPoint[above left](N){$\point N$}

\tkzDrawLine(K,A)
\tkzDrawLine(K,B)
\tkzDrawLine(H,N)
\tkzDrawLine[add=0.2 and 0.5](H,M)

\tkzDrawSegment[style=dashed](H,r)
\tkzLabelSegment[above right](H,r){$r$}

\tkzDrawCircle[R](H,3cm)

\tkzMarkRightAngle[size=0.5](H,N,M)
\tkzMarkAngle[size=0.5](B,K,A)
\tkzLabelAngle(A,K,B){$\gq$}
\tkzMarkAngle[size=0.5](B,M,m)
\tkzLabelAngle(B,M,m){$\gq$}

\end{tikzpicture}
}
A useful optimization at this stage is to determine if the circle intersects the
wedge $\plane KAB$.  If it does not, then there is no hiding and the segment
$\straightline AB$ is entirely visible.  This can happen because the circle is
away from the wedge on the side of $\straightline KA$ or because it is away from
the wedge on the side of $\straightline KB$.  Figure~\ref{figOutside}
illustrates the latter case: note that the circle must not intersect
$\straightline KB$, so the distance between $\point H$ and $\straightline KB$
must be at least $r$; the figure shows the case where the circle is tangent to
$\straightline KB$ at $\point N$, which is the one that interests us in this
section.

Observe that, if $\gq$ is the angle between $\bipoint KA$ and $\bipoint KB$, we
have
\[
\scal{\bipoint KA}{\bipoint KB} = \norm{\bipoint KA} \norm{\bipoint KB}
                                  \cos \gq
\]
Let $\point M$ be the point where $\point H$ projects on $\straightline KB$
parallel to $\straightline KA$.  Since $\ga$ and $\gb$ are the coordinates of
$\bipoint KH$ in the basis $\tuple{\bipoint KA, \bipoint KB}$, we
have
\begin{equation}
{\bipoint HM} = \ga {\bipoint KA} \label{eqnhm1}\text,
\end{equation}
and elementary trigonometry in the triangle $\plane HNM$ yields
\begin{equation}
\scal{\bipoint HM}{\bipoint HM} = \frac{r^2} {\sin^2 \gq} =
                                  \frac{r^2} {1 - \cos^2 \gq}
                                  \label{eqnhm2}\text.
\end{equation}
Eliminating $\bipoint HM$ between equations (\ref{eqnhm1}) and (\ref{eqnhm2}) we
obtain the following conditions on $\ga$ for $\point H$ to be outside the wedge
$\plane KAB$ on the side of $\straightline KB$:
\begin{align*}
\ga &\leq 0\\
\ga^2 &\geq r^2 \frac{\squarenorm{\bipoint KB}}
                     {\pascal{\squarenorm{\bipoint KA}}
                      \pascal{\squarenorm{\bipoint KB}} -
                     {\pascal{\scal{\bipoint KA}{\bipoint KB}}}^2}\text.
\end{align*}
It's straightforward to prove similar conditions on $\gb$ for the circle to be
outside the wedge $\plane KAB$ on the side of $\straightline KA$.

\subsection*{Construction of the cone}
\marginfig[Construction of $\point P$ and $\point P'$ and of the cone.
\label{figP}]{
\begin{tikzpicture}[scale=0.5]
\tkzInit[ymin=-5,ymax=10,xmin=-4,xmax=6]
\tkzClip
\tkzDefPoint(-3,9){A}
\tkzDefPoint(5.5,9){B}
\tkzDefPoint(2,5){H}
\tkzDefPoint(0,-4){K}

\tkzDefTangent[from with R=K](H,3cm) \tkzGetPoints{P'}{P}
\tkzInterLL(K,P)(A,B) \tkzGetPoint{Q}

\tkzDefPointBy[rotation=center H angle -130](P) \tkzGetPoint{r}

\tkzDrawPoints(A,B,H,K,P,P',Q)
\tkzLabelPoint[below left](A){$\point A$}
\tkzLabelPoint[above left](B){$\point B$}
\tkzLabelPoint[below](H){$\point H$}
\tkzLabelPoint[left](K){$\point K$}
\tkzLabelPoint[above left](P){$\point P$}
\tkzLabelPoint[below right](P'){$\point P'$}
\tkzLabelPoint[above right](Q){$\point Q$}

\tkzDrawLine[add=0.05 and 0.05](A,B)
\tkzDrawLine(H,P)
\tkzDrawLine(K,A)
\tkzDrawLine(K,B)
\tkzDrawLine(K,P')
\tkzDrawLine(K,Q)

\tkzDrawSegment[style=dashed](H,r)
\tkzLabelSegment[left](H,r){$r$}

\tkzDrawCircle[R](H,3cm)

\tkzMarkRightAngle[size=0.5](K,P,H)

\end{tikzpicture}
}
Remember that the circle of centre $\point H$ and radius $r$ is the intersection
of the sphere with the plane $\plane KAB$.  As shown on figure~\ref{fig3d},
hiding is determined by the cone of apex $\point K$ tangent to the sphere.  This
cone intersects $\plane KAB$ in two straight lines that go through $\point K$
and are tangent to the circle at points $\point P$ and $\point P'$.  In this
section we are going to determine the coordinates of $\point P$ and $\point P'$
in the basis $\tuple{\bipoint KA, \bipoint KB}$.  Figure~\ref{figP} illustrates
the construction of these points.

$\point P$ is characterized by the two equations:
\begin{align}
\squarenorm{\bipoint PH} &= r^2 \label{eqnp1}\\
\scal{\bipoint PH}{\bipoint KP} &= 0 \label{eqnp2}\text.
\end{align}
Equation (\ref{eqnp2}) may be rewritten as
\[
\scal{\bipoint PH}{\pascal{\bipoint KH + \bipoint HP}} = 0\text,
\]
which yields, when combined with equation (\ref{eqnp1})
\begin{equation}
\scal{\bipoint PH}{\bipoint KH} = r^2 \label{eqnp3}\text.
\end{equation}
Define now $\gg$ and $\gd$ to be the coordinates of $\bipoint PH$ in
$\tuple{\bipoint KA, \bipoint KB}$\[
\bipoint PH = \gg {\bipoint KA} + \gd {\bipoint KB}\text.
\]
Equation (\ref{eqnp3}) is linear in the coordinates of $\point P$ and can be
rewritten as\[
\gg\scal{\bipoint KA}{\bipoint KH} + \gd\scal{\bipoint KB}{\bipoint KH} =
r^2\text.
\]
Note that $\scal{\bipoint KA}{\bipoint KH}$ and
$\scal{\bipoint KB}{\bipoint KH}$ cannot both be $0$ unless $\point K$ is on
$\straightline AB$, so we can either express $\gg$ as a function of $\gd$ or
vice-versa.  If we do the former we obtain\[
\gg = \frac{r^2 - \gd \scal{\bipoint KB}{\bipoint KH}}
           {\scal{\bipoint KA}{\bipoint KH}}\text.
\]
Equation (\ref{eqnp1}) is quadratic in the coordinates of $\point P$ and can be
rewritten as\[
\pa{\gg\bipoint KA + \gd \bipoint KB}^2 = r^2\text.
\]
Pluging the value of $\gg$ above we get
\begin{align*}
\gd^2\pa{
 \pascal{\squarenorm{\bipoint KB}} \pascal{\scal{\bipoint KA}{\bipoint KH}}^2 +
 2\pascal{\scal{\bipoint KA}{\bipoint KB}}
  \pascal{\scal{\bipoint KA}{\bipoint KH}}
  \pascal{\scal{\bipoint KB}{\bipoint KH}} +
 \squarenorm{\bipoint KA} \pascal{\scal{\bipoint KB}{\bipoint KH}}^2} +&\\
2\gd r^2\pa{
 \pascal{\scal{\bipoint KA}{\bipoint KB}}
 \pascal{\scal{\bipoint KA}{\bipoint KH}} -
 \pascal{\squarenorm{\bipoint KA}}
 \pascal{\scal{\bipoint KB}{\bipoint KH}}} +&\\
r^2\pa{
 r^2\pascal{\squarenorm{\bipoint KA}} -
 \pascal{\scal{\bipoint KA}{\bipoint KH}}^2} &= 0\text.
\end{align*}
This equation always has two solutions because the sphere intersects
$\plane KAB$.
\subsection*{Intersection of the cone and the line $\straightline AB$}
Having determined the location of points $\point P$ and $\point P'$ we need to
find the points $\point Q$ and $\point Q'$ where the lines $\straightline KP$
and $\straightline KP'$, respectively, intersect the line $\straightline AB$.
Since $\point Q$ is on $\straightline AB$ there is a $\gl$ such that\[
\bipoint AQ = \gl \bipoint AB\text,
\]
or equivalently
\begin{equation}
\bipoint KQ - \bipoint KA = \gl \bipoint AB \label{eqnq1}\text.
\end{equation}
We can take the scalar product of equation (\ref{eqnq1}) with $\bipoint PH$, and,
noting that $\bipoint KQ$ is orthogonal to $\bipoint PH$ we obtain\[
-\scal{\bipoint KA}{\bipoint PH} = \gl \scal{\bipoint AB}{\bipoint PH}
\text{, or, }
\gl = -\frac{\scal{\bipoint KA}{\bipoint PH}}{\scal{\bipoint AB}{\bipoint PH}}
\text.
\]
Obviously there is a similar equation with $\point P'$ which may be used to
compute $\gl'$, the position of $\point Q'$ on $\straightline AB$.

The point $\point Q$ is in the segment $\straightline AB$ if and only if
$0 \leq \gl \leq 1$.  This does not tell us, however, where $\point Q$ is
located with respect to the cone, to the sphere, or for that matter if it is
in front or behind the camera.  This is what we need to determine next.

\marginfig[Definition of $\point S$ and $\point T$.\label{figST}]{
\begin{tikzpicture}[scale=0.5]
\tkzInit[ymin=-5,ymax=12,xmin=-4,xmax=6]
\tkzClip
\tkzDefPoint(2,11){A}
\tkzDefPoint(5,-4){B}
\tkzDefPoint(0,0){H}
\tkzDefPoint(0,6){K}

\tkzDefTangent[from with R=K](H,3cm) \tkzGetPoints{P2}{P}

\tkzDefLine[orthogonal=through P](K,H) \tkzGetPoint{p}
\tkzInterLL(P,p)(A,B) \tkzGetPoint{T}
\tkzInterLL(P,p)(K,H) \tkzGetPoint{q}

\tkzDefLine[orthogonal=through K](K,H) \tkzGetPoint{k}
\tkzInterLL(K,k)(A,B) \tkzGetPoint{S}

\tkzInterLL(K,P)(A,B) \tkzGetPoint{Q}

\tkzDefPointBy[rotation=center H angle -80](P) \tkzGetPoint{r}

\tkzDrawPoints(A,B,H,K,P,Q,S,T)
\tkzLabelPoint[left](A){$\point A$}
\tkzLabelPoint[left](B){$\point B$}
\tkzLabelPoint[left](H){$\point H$}
\tkzLabelPoint[left](K){$\point K$}
\tkzLabelPoint[below left](P){$\point P$}
\tkzLabelPoint[left](Q){$\point Q$}
\tkzLabelPoint[above right](S){$\point S$}
\tkzLabelPoint[above right](T){$\point T$}

\tkzDrawLine[add=0.2 and 3](A,B)
\tkzDrawLine(H,K)
\tkzDrawLine[add=1 and 1](K,P)
\tkzDrawLine[add=1 and 1](K,P2)
\tkzDrawLine[add=0.05 and 0.2](K,S)
\tkzDrawLine[add=2.2 and 0.2](P,T)

\tkzDrawSegment[style=dashed](H,r)
\tkzLabelSegment[above right](H,r){$r$}

\tkzDrawCircle[R](H,3cm)

\tkzMarkRightAngle[size=0.5](S,K,H)
\tkzMarkRightAngle[size=0.5](P,q,H)

\end{tikzpicture}
}
\subsection*{Location of $\point Q$ and $\point Q'$}
Having computed the values of $\gl$ we need to determine where $\point Q$ is
located with respect to the cone, the sphere and the camera.  To do this we are
going to define two new points, $\point S$ and $\point T$, as shown on
figure~\ref{figST}.  $\point S$ is the intersection of $\straightline AB$ with
the line orthogonal to $\straightline KH$ at $\point K$.  $\point T$ is the
intersection of $\straightline AB$ with the line orthogonal to
$\straightline KH$ at $\point P$.  We are going to find the positions of
$\point S$ and $\point T$ on $\straightline AB$, and, by comparing then with
$\gl$, figure out the location of $\point Q$ with respect to $\point S$ and
$\point T$.

We locate $\point S$ on $\straightline AB$ as follows\[
\bipoint KS = \bipoint KA + \gs \bipoint AB\text.
\]
Taking the scalar product of this equation with $\bipoint KH$, and noting that
$\bipoint KS$ is orthogonal to $\bipoint KH$ we obtain\[
\gs = -\frac{\scal{\bipoint KA}{\bipoint KH}}
            {\scal{\bipoint AB}{\bipoint KH}}
\]
Similarly, we locate $\point T$ on $\straightline AB$ as\[
\bipoint KT = \bipoint KA + \gt \bipoint AB\text.
\]
Since $\bipoint KT = \bipoint KH + \bipoint HP + \bipoint PT$ this can be
written\[
\bipoint PT = \bipoint KA + \gt \bipoint AB - \bipoint KH + \bipoint PH
\]
Again, taking the scalar product of this equation with $\bipoint KH$, and noting
that $\bipoint KT$ is orthogonal to $\bipoint KH$ we obtain\[
\gt = \frac{\scal{\bipoint KH}{\bipoint KH} - \scal{\bipoint PH}{\bipoint KH} -
            \scal{\bipoint KA}{\bipoint KH}}
           {\scal{\bipoint AB}{\bipoint KH}}
\]
We can now determine if $\point Q$ (or alternatively, $\gl$) is an
``interesting'' intersection, \idest, one that intersects the cone behind the
sphere when seen from the camera.  That's because the location of $\point S$ let
us separate what is behind the camera from what is in front of the camera, and
the location of $\point T$ determines where the cone starts (points in front
of $\point T$ cannot be in the cone, although they can be in the sphere).

First, assume that $\point A$ and $\point B$ are in the same order as $\point S$
and $\point T$ on $\straightline AB$.  Then we have $\gs \leq \gt$ and the
intersection $\point Q$ is farther than $\point T$ (as seen from the camera) if
and only if $\gt < \gl$.  $\point Q$ is located on the cone and it is an
interesting intersection (and it participates in hiding) if it falls within the
segment $\straightline AB$.  This is the situation illustrated in
figure~\ref{figST}.

Conversely, if $\point A$ and $\point B$ are in the reverse order
as $\point S$ and $\point T$ on $\straightline AB$ we have $\gt \leq \gs$ and
the intersection is farther than $\point T$ if and only if $\gl < \gt$.

The same analysis must be performed for $\gl'$ corresponding to $\point Q'$.

\subsection*{Point at infinity}
There is another special case that requires some care: if the line 
$\straightline AB$ is in ``hyperbolic'' position, \idest, intersects both halves
of the cone, then one of the values $\gl$, $\gl'$ is smaller than $\gs$ and the
other is greater than $\gs$ (this is the situation shown in figure~\ref{figST}).
Exactly one of $\gl$, $\gl'$ will be retained by the preceding analysis.  To
account for the fact that the points farther than $\point Q$ (such as $\point B$
in figure~\ref{figST}) are inside the cone, we need to add to our analysis an
extra value of $\gl$ equal to an infinity with the sign of $\gt - \gs$.

\marginfig[Intersection with the sphere.\label{figSphere}]{
\begin{tikzpicture}[scale=0.5]
\tkzInit[ymin=-5,ymax=10,xmin=-3.8,xmax=6.2]
\tkzClip
\tkzDefPoint(-3,5){A}
\tkzDefPoint(5.5,5){B}
\tkzDefPoint(2,7){H}
\tkzDefPoint(0,-4){K}

\tkzDefTangent[from with R=K](H,3cm) \tkzGetPoints{P'}{P}
\tkzInterLC[R](A,B)(H,3cm) \tkzGetPoints{Q'}{Q}

\tkzDefPointBy[rotation=center H angle 100](Q') \tkzGetPoint{r}

\tkzDrawPoints(A,B,H,K,P,P',Q,Q')
\tkzLabelPoint[below left](A){$\point A$}
\tkzLabelPoint[below right](B){$\point B$}
\tkzLabelPoint[left](H){$\point H$}
\tkzLabelPoint[left](K){$\point K$}
\tkzLabelPoint[left](P){$\point P$}
\tkzLabelPoint[right](P'){$\point P'$}
\tkzLabelPoint[below left](Q){$\point Q$}
\tkzLabelPoint[above left](Q'){$\point Q'$}

\tkzDrawLine[add=0.05 and 0.05](A,B)
\tkzDrawLine(K,A)
\tkzDrawLine(K,B)
\tkzDrawLine[add=-0.9 and 0.1](K,P)
\tkzDrawLine[add=-0.9 and 0.1](K,P')

\tkzDrawSegment[style=dashed](H,r)
\tkzLabelSegment[left](H,r){$r$}

\tkzDrawCircle[R](H,3cm)

\end{tikzpicture}
}
\subsection*{Intersection with the circle}
To complete the analysis we need to compute the intersection $\point Q$ of the
circle with the line $\straightline AB$. $\point Q$ is on the circle, thus
\begin{equation}
\squarenorm{\bipoint HQ} = r^2 \label{eqnq2}\text;
\end{equation}
it is also on the line $\straightline AB$ and therefore there is a $\gm$ such
that
\[
\bipoint KQ = \bipoint KA + \gm \bipoint AB\text.
\]
We can rewrite $\bipoint HQ$ as follows\[
\bipoint HQ = \bipoint KQ - \bipoint KH =
  \bipoint KA + \gm \bipoint AB - \bipoint KH =
  \bipoint HA + \gm \bipoint AB\text.
\]
Inserting this value in equation (\ref{eqnq2}) we obtain\[
r^2 = \pascal{\bipoint HA + \gm \bipoint AB}^2\text,
\]
meaning that $\gm$ is a solution of\[
\gm \squarenorm{\bipoint AB} + 2 \gm \scal{\bipoint HA}{\bipoint AB}
+ \squarenorm{\bipoint HA} - R^2 = 0\text.
\]
Depending on the location of $\straightline AB$ with respect to the circle,
there can be $0$, $1$, or $2$ intersections.

\subsection*{Concluding the analysis}
If we take the union of the values of $\gl$ and $\gm$ and order them, it is
straightforward to find the visible segments by remembering that
$0 \leq \gl,\gm \leq 1$ for the points that are in the segment
$\straightline AB$.
\end{document}
