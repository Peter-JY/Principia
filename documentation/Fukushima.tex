\documentclass[10pt, a4paper, twoside]{basestyle}

\usepackage[Mathematics]{semtex}
\usepackage{chngcntr}
\counterwithout{equation}{section}

%%%% Shorthands.

%%%% Title and authors.

\title{%
\textdisplay{%
On an Algorithm by Fukushima%
}%
}
\date{\printdate{2019-08-05}}
\author{Pascal~Leroy (phl)}
\begin{document}
\maketitle
\begin{sloppypar}
\noindent
This document describes an algorithm used by Fukushima in his implementation of the complete elliptic integrals of the second kind $B$ and $D$ (\cite{Fukushima2018}).  It
follows the notation and conventions of \cite{Fukushima2011a}, but effectively replaces section 2.2.
\end{sloppypar}

\subsection*{Definitions}
Jacobi's nome $q\of{m}$ is defined as a function of the elliptic integral $K\of{m}$ as:
\[
q\of{m} \DefineAs\exp\of{\frac{-\Pi K\of{1 - m}}{K\of{m}}}
\]
Changing the variable to be $m_c = 1 - m$ and solving for $K\of{m}$ yields:
\[
K\of{m} = \pa{\frac{K\of{m_c}}{\Pi}}\pa{-\log q\of{m_c}}
\]
Now let's split this expression in two terms to separate out the logarithmic part:
\[
\begin{dcases}
X\of{m_c} &\DefineAs -\log q\of{m_c}\\
K_X\of{m_c} &\DefineAs \frac{K\of{m_c}}{\Pi}
\end{dcases}
\]
We have: $K\of{m} = K_X\of{m_c} X\of{m_c}$.

An expression for $E\of{m}$ can be obtained from Legendre's relation (\cite{Fukushima2011a} equation 2.11):
\[
E\of{m} = \pa{1 - \frac{E\of{m_c}}{K\of{m_c}}} K\of{m} + \frac{\Pi}{2 K\of{m_c}}
\]
Similarly, let's split this expression using the terms:
\[
\begin{dcases}
E_X\of{m_c} &\DefineAs \pa{1 - \frac{E\of{m_c}}{K\of{m_c}}}K_X\\
E_0\of{m_c} &\DefineAs \frac{1}{2 K_X\of{m_c}}
\end{dcases}
\]
We have: $E\of{m} = E_X\of{m_c} X\of{m_c} + E_0\of{m_c}$.

\subsection*{Integrals of the second kind for $m$ close to $1$}
Fukushima defines $B\of{m}$ as:
\[
B\of{m} \DefineAs \frac{E\of{m} - m_c K\of{m}}{m}
\]
This expression can be rewritten using the definitions above:
\begin{align*}
B\of{m} &= \frac{1}{m}\pa{E_X\of{m_c} X\of{m_c} + E_0\of{m_c} - m_c K_X\of{m_c} X\of{m_c}}\\
&= \frac{1}{m}\pa{X\of{m_c}\pa{E_X\of{m_c} - m_c K_X\of{m_c}} + \frac{1}{2 K_X\of{m_c}}}
\end{align*}
Similarly the definition of $D\of{m}$:
\[
D\of{m} \DefineAs \frac{K\of{m} - E\of{m}}{m}
\]
can be rewritten as:
\begin{align*}
D\of{m} &= \frac{1}{m}\pa{K_X\of{m_c} X\of{m_c} - E_X\of{m_c} X\of{m_c} - E_0\of{m_c}}\\
&= \frac{1}{m}\pa{X\of{m_c}\pa{K_X\of{m_c} - E_X\of{m_c}} - \frac{1}{2 K_X\of{m_c}}}
\end{align*}
These formulæ provide a means to compute $B\of{m}$ and $D\of{m}$ for $m$ close to $1$.  First, a polynomial approximation of $q\of{m_c}$
is computed, whose leading term is of order $m_c / 16$.  Then the $\log$ of that approximation is evaluated, yielding $X\of{m_c}$ (this
is the part that carries the logarithmic singularity).  Finally, $E_X\of{m_c}$ and $K_X\of{m_c}$ are computed using
Taylor or Maclaurin approximations.

It is easy to see that $B\of{m} = X\of{m_c} K_X\of{m_c} - D\of{m}$, which provides a simpler formula for computing $B\of{m}$ once 
$D\of{m}$ is known.

\subsection*{Integrals of the second kind when $m$ tends towards $1$}
We are now interested in computing the leading term of $B\of{m}$ and $D\of{m}$ when $\conv m 1$.  First, we have $B\of{1} = 1$ 
(\cite{Fukushima2011a}, equation 1.9).  However, $\conv {D\of{m}} +\infty$ when $\conv m 1$.  To deal with this singularity we recall that:
\[
B\of{m} = X\of{m_c} K_X\of{m_c} - D\of{m}
\]
thus:
\[
D\of{m} = X\of{m_c} K_X\of{m_c} - B\of{m}
\]
Remember that $X\of{m_c} = -\log q\of{m_c}$ and that $q\of{m_c} = m_c / 16 + \BigO\of{m_c^2}$.  Therefore:
\[
X\of{m_c} = \log 16 - \log\of{m_c} + \BigO\of{m_c^2}
\]
Furthermore $K\of{0} = \Pi / 2$, so $K_X\of{0} = 1/2$.  Putting all these relations together we obtain the following equation:
\[
D\of{m} = 2 \log 2 - 1 - \frac{\log m_c}{2} + \BigO\of{m_c^2}
\]

\subsection*{Integrals of the second kind for $m$ close to $0$}
The expressions defined in the first section can be rewritten by changing the variable to be $m = 1 - m_c$.  In particular:
\begin{align*}
E_X\of{m} &= \pa{1 - \frac{E\of{m}}{K\of{m}}K_X\of{m}}\\
&= \frac{1}{\Pi}\pa{K\of{m} - E\of{m}}
\end{align*}
Thus:
\[
D\of{m} = \frac{\Pi}{m}E_X\of{m}
\]
Similarly, define $B_X^*\of{m}$ as follows:
\[
B_X^*\of{m} \DefineAs E_X\of{m} - m K_X\of{m}
\]
We have:
\begin{align*}
B_X^*\of{m} &= \frac{1}{\Pi}\pa{K\of{m} - E\of{m} - m K\of{m}}\\
&= \frac{1}{\Pi}\pa{m_c K\of{m} - E\of{m}}\\
&= -\frac{m B\of{m}}{\Pi}
\end{align*}
Therefore:
\[
B\of{m} = -\frac{\Pi}{m} B_X^*\of{m}
\]
These formulæ make it possible, by computing a Maclaurin approximation of $B_X^*\of{m}$ and $E_X\of{m}$, to evaluate $B\of{m}$
and $D\of{m}$ for $m$ close to $0$.

\subsection*{Conclusion}
We have demonstrated how \cite{Fukushima2018} uses different techniques from the ones detailed in \cite{Fukushima2011a} in order to handle the 
logarithmic singularities of the $B$ and $D$ complete integrals of the second kind: while \cite{Fukushima2011a} divides the leading logarithmic term $\log \frac{m_c}{16}$, \cite{Fukushima2018} divides the complete logarithmic term $\log q\of{m_c}$.
\end{document}